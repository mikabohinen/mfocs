\documentclass[a4paper]{article}
% Some basic packages
\usepackage[utf8]{inputenc}
\usepackage[T1]{fontenc}
\usepackage{textcomp}
\usepackage[english]{babel}
\usepackage{url}
\usepackage{graphicx}
\usepackage{float}
\usepackage{booktabs}
\usepackage{enumitem}
\usepackage{tikz-cd}

\pdfminorversion=7

% Don't indent paragraphs, leave some space between them
\usepackage{parskip}

% Hide page number when page is empty
\usepackage{emptypage}
\usepackage{subcaption}
\usepackage{multicol}
\usepackage{xcolor}

% Math stuff
\usepackage{amsmath, amsfonts, mathtools, amsthm, amssymb}
\usepackage{mathrsfs}
\usepackage{cancel}
\usepackage{bm}
\newcommand\N{\ensuremath{\mathbb{N}}}
\newcommand\R{\ensuremath{\mathbb{R}}}
\newcommand\Z{\ensuremath{\mathbb{Z}}}
\renewcommand\O{\ensuremath{\emptyset}}
\newcommand\Q{\ensuremath{\mathbb{Q}}}
\newcommand\C{\ensuremath{\mathbb{C}}}

\usepackage{systeme}

\let\svlim\lim\def\lim{\svlim\limits}

\let\implies\Rightarrow
\let\impliedby\Leftarrow
\let\iff\Leftrightarrow
\let\epsilon\varepsilon

\usepackage{stmaryrd}
\newcommand\contra{\scalebox{1.5}{$\lightning$}}

\definecolor{correct}{HTML}{009900}
\newcommand\correct[2]{\ensuremath{\:}{\color{red}{#1}}\ensuremath{\to }{\color{correct}{#2}}\ensuremath{\:}}
\newcommand\green[1]{{\color{correct}{#1}}}

\newcommand\hr{
    \noindent\rule[0.5ex]{\linewidth}{0.5pt}
}

\newcommand\hide[1]{}

\usepackage{siunitx}
\sisetup{locale = US}

\usepackage{mdframed}
\mdfsetup{skipabove=1em,skipbelow=0em}
\theoremstyle{definition}
\newmdtheoremenv[nobreak=true]{definition}{Definition}
\newmdtheoremenv[nobreak=true]{property}{Property}
\newmdtheoremenv[nobreak=true]{corollary}{Corollary}
\newmdtheoremenv[nobreak=true]{theorem}{Theorem}
\newmdtheoremenv[nobreak=true]{lemma}{Lemma}
\newmdtheoremenv[nobreak=true]{proposition}{Proposition}

\newtheorem*{example}{Example}
\newtheorem*{notation}{Notation}
\newtheorem*{remark}{Remark}
\newtheorem*{note}{Note}
\newtheorem*{problem}{Problem}
\newtheorem*{observe}{Observe}
\newtheorem*{intuition}{Intuition}
% ... (add other unnumbered theorem-like environments as per your requirement)

\usepackage{etoolbox}
\AtEndEnvironment{example}{\null\hfill$\diamond$}%
% ... (add other environments to end with a diamond if required)

\makeatletter
\def\thm@space@setup{%
  \thm@preskip=\parskip \thm@postskip=0pt
}

\newcommand{\exercise}[1]{%
    \def\@exercise{#1}%
    \subsection*{Exercise #1}
}

\newcommand{\subexercise}[1]{%
    \subsubsection*{Exercise \@exercise.#1}
}

\usepackage{xifthen}
\def\testdateparts#1{\dateparts#1\relax}
\def\dateparts#1 #2 #3 #4 #5\relax{
    \marginpar{\small\textsf{\mbox{#1 #2 #3 #5}}}
}

\def\@lecture{}%
\newcommand{\lecture}[3]{
    \ifthenelse{\isempty{#3}}{%
        \def\@lecture{Lecture #1}%
    }{%
        \def\@lecture{Lecture #1: #3}%
    }%
    \subsection*{\@lecture}
    \marginpar{\small\textsf{\mbox{#2}}}
}

\usepackage{fancyhdr}
\pagestyle{fancy}

\fancyhead[RO,LE]{\@lecture}
\fancyfoot[RO,LE]{\thepage}
\fancyfoot[C]{\leftmark}

\makeatother

\usepackage{todonotes}
\usepackage{tcolorbox}

\tcbuselibrary{breakable}

\newenvironment{correction}{\begin{tcolorbox}[
    arc=0mm,
    colback=white,
    colframe=green!60!black,
    title=Remark,
    fonttitle=\sffamily,
    breakable
]}{\end{tcolorbox}}

\newenvironment{notebox}[1]{\begin{tcolorbox}[
    arc=0mm,
    colback=white,
    colframe=white!60!black,
    title=#1,
    fonttitle=\sffamily,
    breakable
]}{\end{tcolorbox}}

\usepackage{import}
\usepackage{pdfpages}
\usepackage{transparent}
\newcommand{\incfig}[1]{%
    \def\svgwidth{\columnwidth}
    \import{./figures/}{#1.pdf_tex}
}

\pdfsuppresswarningpagegroup=1

\author{Mika Bohinen}

\title{Sheet 1}
\begin{document}
\maketitle
\begin{exercise}{2}
  \begin{enumerate}[label=(\alph*)]
    \item Let $ F \in \text{Fun}(*/G_1, */G_2) $. We then have
      that $ F(*_{G_1}) = *_{G_2} $. Moreover the functoriality of $ F $ means that if $ g, g' \in G_1 $ then $ F(g' \cdot g) = F(g')\cdot F(g) $. Hence $ F $ is just a group homomorphism.

      Now, if $ \eta: F \implies H $ is a natural transformation for $ F,H \in \text{Fun}(*/G_1, */G_2) $ and $ g_1 \in G_1 $, then we get the following commutative diagram
      \[\begin{tikzcd}
	      {*_{G_2}} & {*_{G_2}} \\
	      {*_{G_2}} & {*_{G_2}}
	      \arrow["Fg_1", from=1-1, to=1-2]
	      \arrow["{\eta_*}"', from=1-1, to=2-1]
	      \arrow["{\eta_*}", from=1-2, to=2-2]
	      \arrow["Hg_1"', from=2-1, to=2-2]
      \end{tikzcd}\]
      We then have that $ \eta_* = g_2 $ for some $ g_2 \in G_2  $. The commutativity of the diagram tells us that
      \begin{equation*}
      Fg_1 = g_2^{-1} H(g_1) g_2
      .\end{equation*}

      Thus a natural transformation is just conjugation by an element of the group $ G_2 $.

    \item Let $ F \in \text{Fun}(*/\mathbb{Z}, */G) $. Then, by the previous exercise we know that $ \text{Fun}(*/\mathbb{Z}, */G) \cong \text{Hom}_{\mathbf{Grp}}(\mathbb{Z}, G) $. Hence $ F $ corresponds to some $ f \in \text{Hom}_\mathbf{Grp}(\mathbb{Z}, G) $. Moreover, since $ f(0) = e_G $ and $ f(n) = f(1)^{n} $ we have that the image of $ f $ is simply the cyclic subgroup $ \left\langle f(1) \right\rangle $. The most natural thing is therefore to associate $ f \in \text{Hom}_{\mathbf{Grp}}(\mathbb{Z}, G) $ with $ f(1) $. As $ f(1) $ can be any element in $ G $ we therefore have that $ \text{Hom}_\mathbf{Grp}(\mathbb{Z}, G) \cong G $ as a group isomorphism (the isomorphism being given by $ f \mapsto f(1) $).

      With $ n \in \mathbb{Z} $ considered as a morphism and $ a \in G $ which represents the natural transformation $ \eta: F \implies H $, and $ F(n) = g_1^{n} $, $ H(n) = g_2^{n} $
      we have the equation
      \begin{equation*}
      g_1^{n} = a^{-1}g_2^{n}a
      .\end{equation*}

      This has to hold for all $ n $ but it suffices that it holds for $ n = 1 $ where we have the equation
      \begin{equation*}
      g_1 = a^{-1}g_2 a
      .\end{equation*}

      Hence natural transformations $ F \implies G $ are in one-to-one correspondence with conjugation relations of $ g_1 $ with $ g_2 $.
    \end{enumerate}
  \end{exercise}

  \begin{exercise}{3}
    As $ \text{Hom}_{\text{Vect}_k^{fd}}(-, k) $ is contravariant we cannot technically have an equivalence of categories. However, we can speak of a duality of categories where we just require that both $ F $ and $ G $ are contravariant.

    We first note that equivalence of categories is an equivalence relation. Considering the diagram
    \[\begin{tikzcd}
	    {\text{Vect}_k^{fd}} & {(\text{Vect}_k^{fd})^{\text{op}}} \\
	    {\text{Mat}_k} & {(\text{Mat}_k)^\text{op}}
	    \arrow[from=1-1, to=1-2]
	    \arrow[from=1-1, to=2-1]
	    \arrow[from=1-2, to=2-2]
	    \arrow[from=2-1, to=2-2]
    \end{tikzcd}\]
    it suffices to show that $ \text{Mat}_k \to (\text{Mat}_k)^{\text{op}} $ is a duality of categories since we know that the two vertical arrows already are part of an equivalence of categories.

    The contravariant functor $ \text{Mat}_k \to (\text{Mat}_k)^{\text{op}} $ is the identity on the objects and sends each matrix to its transpose. Hence if we create the contravariant functor $ (\text{Mat}_k)^{\text{op}} \to \text{Mat}_k $ which does exactly the same then their composition is the identity so that we have a contravariant isomorphism of categories.

    Thus the map $ \text{Vect}_k^{fd} \to  (\text{Vect}_k^{fd})^{\text{op}} $ is a duality of categories.
  \end{exercise}

  \begin{exercise}{4}
    Let $ f,g \in \text{Hom}(x, y) $. Then, since $ \text{End}(x) $ is the trivial group, we must have that $ g^{-1}\circ f = 1_x $. This implies that $ f = g $ showing that $ |\text{Hom}(x,y)| = 1 $.

    Next, let $ F: \mathcal{C} \to * $ be the unique functor from $ \mathcal{C} $ to $ * $ (as $ * $ is terminal in \textbf{Cat}). We then just have to show that $ F $ is fully-faithful and essentially surjective. As $ F $ is surjective on objects, it must be essentially surjective. We therefore only need to show that $ F $ is fully-faithful. This is quite straight-forward:

    \begin{enumerate}
      \item \textbf{Injectivity}. Let $ f,g \in \text{Hom}(x,y) $ such that $ Ff = Fg $. Since $ \text{Hom}(x,y) = \{1\} $ we must have that $ f = g $.

      \item \textbf{Surjectivity}. For $ x, y \in \mathcal{C} $ we have that $ F(x) = * = F(y) $. Hence $ \text{Hom}(Fx, Fy) = \{1_*\} $. As $ |\text{Hom}(x,y)| = 1 $ this unique isomorphism between $ x $ and $ y $ must get mapped to $ 1_* $ which shows that $ F: \text{Hom}(x,y) \to \text{Hom}(Fx,Fy) $ is surjective.
    \end{enumerate}

    Since $ F $ is both fully-faithful and essentially surjective we can conclude that there exists a $ G: * \to \mathcal{C} $ such that $ GF \cong \text{id}_\mathcal{C} $ and $ FG \cong \text{id}_* $ which shows that $ \mathcal{C} $ is equivalent to the discrete category with a single object.
  \end{exercise}
  \end{document}
