\documentclass[a4paper]{article}
% Some basic packages
\usepackage[utf8]{inputenc}
\usepackage[T1]{fontenc}
\usepackage{textcomp}
\usepackage[english]{babel}
\usepackage{url}
\usepackage{graphicx}
\usepackage{float}
\usepackage{booktabs}
\usepackage{enumitem}
\usepackage{tikz-cd}

\pdfminorversion=7

% Don't indent paragraphs, leave some space between them
\usepackage{parskip}

% Hide page number when page is empty
\usepackage{emptypage}
\usepackage{subcaption}
\usepackage{multicol}
\usepackage{xcolor}

% Math stuff
\usepackage{amsmath, amsfonts, mathtools, amsthm, amssymb}
\usepackage{mathrsfs}
\usepackage{cancel}
\usepackage{bm}
\newcommand\N{\ensuremath{\mathbb{N}}}
\newcommand\R{\ensuremath{\mathbb{R}}}
\newcommand\Z{\ensuremath{\mathbb{Z}}}
\renewcommand\O{\ensuremath{\emptyset}}
\newcommand\Q{\ensuremath{\mathbb{Q}}}
\newcommand\C{\ensuremath{\mathbb{C}}}

\usepackage{systeme}

\let\svlim\lim\def\lim{\svlim\limits}

\let\implies\Rightarrow
\let\impliedby\Leftarrow
\let\iff\Leftrightarrow
\let\epsilon\varepsilon

\usepackage{stmaryrd}
\newcommand\contra{\scalebox{1.5}{$\lightning$}}

\definecolor{correct}{HTML}{009900}
\newcommand\correct[2]{\ensuremath{\:}{\color{red}{#1}}\ensuremath{\to }{\color{correct}{#2}}\ensuremath{\:}}
\newcommand\green[1]{{\color{correct}{#1}}}

\newcommand\hr{
    \noindent\rule[0.5ex]{\linewidth}{0.5pt}
}

\newcommand\hide[1]{}

\usepackage{siunitx}
\sisetup{locale = US}

\usepackage{mdframed}
\mdfsetup{skipabove=1em,skipbelow=0em}
\theoremstyle{definition}
\newmdtheoremenv[nobreak=true]{definition}{Definition}
\newmdtheoremenv[nobreak=true]{property}{Property}
\newmdtheoremenv[nobreak=true]{corollary}{Corollary}
\newmdtheoremenv[nobreak=true]{theorem}{Theorem}
\newmdtheoremenv[nobreak=true]{lemma}{Lemma}
\newmdtheoremenv[nobreak=true]{proposition}{Proposition}

\newtheorem*{example}{Example}
\newtheorem*{notation}{Notation}
\newtheorem*{remark}{Remark}
\newtheorem*{note}{Note}
\newtheorem*{problem}{Problem}
\newtheorem*{observe}{Observe}
\newtheorem*{intuition}{Intuition}
% ... (add other unnumbered theorem-like environments as per your requirement)

\usepackage{etoolbox}
\AtEndEnvironment{example}{\null\hfill$\diamond$}%
% ... (add other environments to end with a diamond if required)

\makeatletter
\def\thm@space@setup{%
  \thm@preskip=\parskip \thm@postskip=0pt
}

\newcommand{\exercise}[1]{%
    \def\@exercise{#1}%
    \subsection*{Exercise #1}
}

\newcommand{\subexercise}[1]{%
    \subsubsection*{Exercise \@exercise.#1}
}

\usepackage{xifthen}
\def\testdateparts#1{\dateparts#1\relax}
\def\dateparts#1 #2 #3 #4 #5\relax{
    \marginpar{\small\textsf{\mbox{#1 #2 #3 #5}}}
}

\def\@lecture{}%
\newcommand{\lecture}[3]{
    \ifthenelse{\isempty{#3}}{%
        \def\@lecture{Lecture #1}%
    }{%
        \def\@lecture{Lecture #1: #3}%
    }%
    \subsection*{\@lecture}
    \marginpar{\small\textsf{\mbox{#2}}}
}

\usepackage{fancyhdr}
\pagestyle{fancy}

\fancyhead[RO,LE]{\@lecture}
\fancyfoot[RO,LE]{\thepage}
\fancyfoot[C]{\leftmark}

\makeatother

\usepackage{todonotes}
\usepackage{tcolorbox}

\tcbuselibrary{breakable}

\newenvironment{correction}{\begin{tcolorbox}[
    arc=0mm,
    colback=white,
    colframe=green!60!black,
    title=Remark,
    fonttitle=\sffamily,
    breakable
]}{\end{tcolorbox}}

\newenvironment{notebox}[1]{\begin{tcolorbox}[
    arc=0mm,
    colback=white,
    colframe=white!60!black,
    title=#1,
    fonttitle=\sffamily,
    breakable
]}{\end{tcolorbox}}

\usepackage{import}
\usepackage{pdfpages}
\usepackage{transparent}
\newcommand{\incfig}[1]{%
    \def\svgwidth{\columnwidth}
    \import{./figures/}{#1.pdf_tex}
}

\pdfsuppresswarningpagegroup=1

\author{Mika Bohinen}

\title{Homological Algebra \\ Sheet 3}
\begin{document}
\maketitle
\begin{exercise}{2}
  \begin{enumerate}[label=(\roman*)]
    \item Consider the following free resolution of $ k[x]/(x-a) $
      \begin{equation*}
        0 \to k[x] \xrightarrow{\cdot (x - a)} k[x] \to \frac{k[x]}{(x-a)} \to 0.
      \end{equation*}
      Tensoring with $ k[x]/(x - b) $ over $ k[x] $ gives us the following complex
      \begin{equation*}
        0 \to \frac{k[x]}{(x-b)} \xrightarrow{ \cdot (b - a)} k[x]/(x - b) \to 0
      .\end{equation*}
      Denoting the multiplication map $ \cdot (x - a) $ by $ f $ there are two cases two consider.

      ($ a \neq b $): In this case we have that
      \begin{equation*}
        \text{ker}(f) = \{c \in k \mid c(x - a) = c(b - a) = 0\}.
      \end{equation*}
      Hence, if $ c \in \text{ker}(f) $ we have that $ c b = ca $ so we must have $ c = 0 $ as in the other case we get $ b = a $ which is a contradiction.
      \begin{align*}
        \text{Tor}_1^{k[x]} \left( \frac{k[x]}{(x-a)}, \frac{k[x]}{(x - b)} \right) &= \text{ker}(f) = 0 \\
        \text{Tor}_0^{k[x]} \left( \frac{k[x]}{(x-a)}, \frac{k[x]}{(x - b)} \right) &= \frac{k[x]}{(x- a)} \otimes_{k[x]} \frac{k[x]}{(x - b)} = \frac{k[x]}{(x-a,x-b)} = 0.
      \end{align*}

      ($ a = b $): In this case we have that
      \begin{equation*}
        \text{ker}(f) = \{c \in k \mid c(x - a) = c(b - a) = 0\}.
      \end{equation*}
      However this requirement for being in $ \text{ker}(f) $ holds true for every $ c \in k $ and hence $ \text{ker}(f) = k[x]/(x - b) $. Thus
      \begin{align*}
        \text{Tor}_1^{k[x]}\left( \frac{k[x]}{(x-a)}, \frac{k[x]}{(x - b)} \right) &= \text{ker}(f) = \frac{k[x]}{(x - b)} \\
        \text{Tor}_0^{k[x]} \left( \frac{k[x]}{(x-a)}, \frac{k[x]}{(x - b)} \right) &= \frac{k[x]}{(x- a)} \otimes_{k[x]} \frac{k[x]}{(x - b)} = \frac{k[x]}{(x-b,x-b)} = \frac{k[x]}{(x - b)}
      .\end{align*}

    \item Take the free resolution
      \begin{equation*}
        0 \to \mathbb{Z} \xrightarrow{\cdot a} \mathbb{Z} \to \frac{\mathbb{Z}}{(a)} \to 0
      .\end{equation*}
      Tensoring with $ \mathbb{Z}/b $ we get the complex
      \begin{equation*}
        0 \to \frac{\mathbb{Z}}{(b)} \xrightarrow{f} \frac{\mathbb{Z}}{(b)} \to 0
      \end{equation*}
      with $ f(1) = a \text{ mod } b $. We then have that
      \begin{equation*}
        \text{ker}(f) = \{c \in \mathbb{Z}/(a) \mid b | ac\} = \mathbb{Z}/(d)
      \end{equation*}
      where $ d = \text{gcd}(a,b) $. Hence
      \begin{align*}
        \text{Tor}_1^{\mathbb{Z}} \left( \frac{\mathbb{Z}}{(a)}, \frac{\mathbb{Z}}{(b)} \right) &= \text{ker}(f) = \frac{\mathbb{Z}}{(d)} \\
        \text{Tor}_0^{\mathbb{Z}} \left( \frac{\mathbb{Z}}{(a)}, \frac{\mathbb{Z}}{(b)} \right) &= \text{coker}(f) = \frac{\mathbb{Z}}{(a)} \otimes_{\mathbb{Z}} \frac{\mathbb{Z}}{(b)} = \frac{\mathbb{Z}}{(d)}
      .\end{align*}

    \item Take the following free resolution of $ \mathbb{Z}/(2) $ over $ \mathbb{Z}/(4) $
      \begin{equation*}
        \cdots \xrightarrow{\cdot 2} \frac{\mathbb{Z}}{(4)} \xrightarrow{\cdot 2} \frac{\mathbb{Z}}{(4)} \to \frac{\mathbb{Z}}{(2)} \to 0
      .\end{equation*}
      Applying the hom functor $ \text{Hom}_{\mathbb{Z}/(4)}(-, \mathbb{Z}/(2)) $ we get the following complex
      \begin{equation*}
        0 \to \text{Hom}_{\mathbb{Z}/(4)}(\mathbb{Z}/(4), \mathbb{Z}/(2)) \xrightarrow{\cdot 2} \text{Hom}_{\mathbb{Z}/(4)}(\mathbb{Z}/(4), \mathbb{Z}/(2)) \xrightarrow{\cdot 2} \cdots
      \end{equation*}
      which is equivalent to
      \begin{equation*}
        0 \to \frac{\mathbb{Z}}{(2)} \xrightarrow{\cdot 0} \frac{\mathbb{Z}}{(2)} \xrightarrow{\cdot 0} \cdots
      \end{equation*}
      so that
      \begin{equation*}
        \text{Ext}_{\mathbb{Z}/(4)}^{*} = \bigoplus_{i = 0}^{\infty} \frac{\mathbb{Z}}{(2)}
      .\end{equation*}

    \item Take the following free resolution of $ \mathbb{Z}/(2^{b}) $
      \begin{equation*}
      \cdots \xrightarrow{\cdot 2^{a-b}} \mathbb{Z}/(2^{a}) \xrightarrow{\cdot 2^{b}} \mathbb{Z}/(2^{a}) \xrightarrow{\cdot 2^{a - b}} \mathbb{Z}/(2^{a}) \xrightarrow{\cdot 2^{b}} \mathbb{Z}/(2^{a}) \to \mathbb{Z}/(2^{b}) \to 0
      .\end{equation*}
      Applying $ \text{Hom}_{\mathbb{Z}/(2^{a})} (-, \mathbb{Z}/(2^{c})) $ we get the following complex
      \begin{equation*}
        0 \to \mathbb{Z}/(2^{c}) \xrightarrow{\cdot 0} \mathbb{Z}/(2^{c}) \xrightarrow{\cdot 2^{a - b}} \mathbb{Z}/(2^{c}) \xrightarrow{\cdot 0} \cdots
      .\end{equation*}
      If $ a - b\geq c $ then $ \cdot 2^{a - b} = \cdot 0 $ so that
      \begin{equation*}
        \text{Ext}^{*}_{\mathbb{Z}/(2^{a})}(\mathbb{Z}/(2^{b}), \mathbb{Z}/(2^{c})) = \bigoplus_{ i = 0}^{\infty} \mathbb{Z}/(2^{c})
      .\end{equation*}
      On the other hand, if $ c > a - b $ then we have that
      \begin{align*}
        \text{ker}(f) &= \{n \in \mathbb{Z}/(2^{c}) \mid 2^{c} | 2^{a- b}n\} \\
                      &= \{n \in \mathbb{Z}/(2^{c}) \mid 2^{c + b - a} | n\} \\
                      &= (2^{c + b - a})\mathbb{Z}/2^{c}\mathbb{Z} \\
                      &=
      .\end{align*}

    \item
  \end{enumerate}
\end{exercise}
\end{document}
