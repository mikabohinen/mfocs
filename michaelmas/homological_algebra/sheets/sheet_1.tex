\documentclass[a4paper]{article}
% Some basic packages
\usepackage[utf8]{inputenc}
\usepackage[T1]{fontenc}
\usepackage{textcomp}
\usepackage[english]{babel}
\usepackage{url}
\usepackage{graphicx}
\usepackage{float}
\usepackage{booktabs}
\usepackage{enumitem}
\usepackage{tikz-cd}

\pdfminorversion=7

% Don't indent paragraphs, leave some space between them
\usepackage{parskip}

% Hide page number when page is empty
\usepackage{emptypage}
\usepackage{subcaption}
\usepackage{multicol}
\usepackage{xcolor}

% Math stuff
\usepackage{amsmath, amsfonts, mathtools, amsthm, amssymb}
\usepackage{mathrsfs}
\usepackage{cancel}
\usepackage{bm}
\newcommand\N{\ensuremath{\mathbb{N}}}
\newcommand\R{\ensuremath{\mathbb{R}}}
\newcommand\Z{\ensuremath{\mathbb{Z}}}
\renewcommand\O{\ensuremath{\emptyset}}
\newcommand\Q{\ensuremath{\mathbb{Q}}}
\newcommand\C{\ensuremath{\mathbb{C}}}

\usepackage{systeme}

\let\svlim\lim\def\lim{\svlim\limits}

\let\implies\Rightarrow
\let\impliedby\Leftarrow
\let\iff\Leftrightarrow
\let\epsilon\varepsilon

\usepackage{stmaryrd}
\newcommand\contra{\scalebox{1.5}{$\lightning$}}

\definecolor{correct}{HTML}{009900}
\newcommand\correct[2]{\ensuremath{\:}{\color{red}{#1}}\ensuremath{\to }{\color{correct}{#2}}\ensuremath{\:}}
\newcommand\green[1]{{\color{correct}{#1}}}

\newcommand\hr{
    \noindent\rule[0.5ex]{\linewidth}{0.5pt}
}

\newcommand\hide[1]{}

\usepackage{siunitx}
\sisetup{locale = US}

\usepackage{mdframed}
\mdfsetup{skipabove=1em,skipbelow=0em}
\theoremstyle{definition}
\newmdtheoremenv[nobreak=true]{definition}{Definition}
\newmdtheoremenv[nobreak=true]{property}{Property}
\newmdtheoremenv[nobreak=true]{corollary}{Corollary}
\newmdtheoremenv[nobreak=true]{theorem}{Theorem}
\newmdtheoremenv[nobreak=true]{lemma}{Lemma}
\newmdtheoremenv[nobreak=true]{proposition}{Proposition}

\newtheorem*{example}{Example}
\newtheorem*{notation}{Notation}
\newtheorem*{remark}{Remark}
\newtheorem*{note}{Note}
\newtheorem*{problem}{Problem}
\newtheorem*{observe}{Observe}
\newtheorem*{intuition}{Intuition}
% ... (add other unnumbered theorem-like environments as per your requirement)

\usepackage{etoolbox}
\AtEndEnvironment{example}{\null\hfill$\diamond$}%
% ... (add other environments to end with a diamond if required)

\makeatletter
\def\thm@space@setup{%
  \thm@preskip=\parskip \thm@postskip=0pt
}

\newcommand{\exercise}[1]{%
    \def\@exercise{#1}%
    \subsection*{Exercise #1}
}

\newcommand{\subexercise}[1]{%
    \subsubsection*{Exercise \@exercise.#1}
}

\usepackage{xifthen}
\def\testdateparts#1{\dateparts#1\relax}
\def\dateparts#1 #2 #3 #4 #5\relax{
    \marginpar{\small\textsf{\mbox{#1 #2 #3 #5}}}
}

\def\@lecture{}%
\newcommand{\lecture}[3]{
    \ifthenelse{\isempty{#3}}{%
        \def\@lecture{Lecture #1}%
    }{%
        \def\@lecture{Lecture #1: #3}%
    }%
    \subsection*{\@lecture}
    \marginpar{\small\textsf{\mbox{#2}}}
}

\usepackage{fancyhdr}
\pagestyle{fancy}

\fancyhead[RO,LE]{\@lecture}
\fancyfoot[RO,LE]{\thepage}
\fancyfoot[C]{\leftmark}

\makeatother

\usepackage{todonotes}
\usepackage{tcolorbox}

\tcbuselibrary{breakable}

\newenvironment{correction}{\begin{tcolorbox}[
    arc=0mm,
    colback=white,
    colframe=green!60!black,
    title=Remark,
    fonttitle=\sffamily,
    breakable
]}{\end{tcolorbox}}

\newenvironment{notebox}[1]{\begin{tcolorbox}[
    arc=0mm,
    colback=white,
    colframe=white!60!black,
    title=#1,
    fonttitle=\sffamily,
    breakable
]}{\end{tcolorbox}}

\usepackage{import}
\usepackage{pdfpages}
\usepackage{transparent}
\newcommand{\incfig}[1]{%
    \def\svgwidth{\columnwidth}
    \import{./figures/}{#1.pdf_tex}
}

\pdfsuppresswarningpagegroup=1

\author{Mika Bohinen}

\title{Homological Algebra \\ Sheet 1}
\begin{document}
\maketitle

\begin{exercise}{3}
  Since it wasn't specified we assume that the horizontal sequences are exact and that the diagram commutes.

  The result we want to prove follows if we can show that $ \text{ker}(j) = 0 = \text{coker}(j) $.

  From the snake lemma we know that there exist an exact sequence of the form
  \begin{equation*}
    0 \to \text{ker }i \to \text{ker }j \to \text{ker }k \xrightarrow{\delta} \text{coker }i \to \text{coker } j \to \text{coker }k \to 0.
  \end{equation*}
  Since $ i $ and $ k $ are isomorphisms, and the fact that we are in an abelian category, we have that this exact sequence turns into
  \begin{equation*}
    0 \to 0 \to \text{ker }j \to 0 \xrightarrow{\delta} 0 \to \text{coker }j \to 0 \to 0
  .\end{equation*}
  Hence we must have that $ \text{ker }j = 0 = \text{coker } j $ and so $ j $ is an isomorphism.
\end{exercise}

\begin{exercise}{4}
  Let the middle module be given by
  \begin{equation*}
    M = \frac{k[x,y,z] \oplus k[x,y,z]}{\left\langle (x^{2}, 0), (xz, 0), (z^{3}, 0) \right\rangle}
  .\end{equation*}
  Then the sequence
  \begin{equation*}
  0 \to M_1 \to M \to M_2 \to 0
  \end{equation*}
  is exact. Moreover, there does not exist a section $ s: M_2 \to M $ and hence the sequence does not split.
\end{exercise}

\begin{exercise}{5}
  This follows from showing that $ \mathbb{Z} $ is a projective $ \mathbb{Z} $-module and $ \mathbb{Q} $ is an injective $ \mathbb{Z} $-module. More generally, it is true that any free $ \mathbb{Z} $-module is projective (or for that matter any free $ R $-module).

  The case for just $ \mathbb{Z} $ is straightforward. Since $ p:B \to \mathbb{Z} $ is surjective there exist a $ b \in B $ such that $ p(b) = 1$. Let $ s: \mathbb{Z} \to B $ be given by
  $ s(n) = ns(1)= n\cdot b $. Then
  \begin{equation*}
    (p\circ s)(n) = p(n\cdot b) = n \cdot p(b) = n
  .\end{equation*}
  Hence $ p \circ s = \text{id}_{\mathbb{Z}} $ so that the sequence must split by the splitting lemma.

  To show that $ 0 \to \mathbb{Q} \to B \to C \to 0 $ splits it's easiest to show that $ \mathbb{Q} $ is an injective $ \mathbb{Z} $-module because then the injectivity of $ \mathbb{Q} $ tells us that $ \text{Hom}(-, \mathbb{Q}) $ is exact and so we get that there is an $ r \in \text{Hom}(B, \mathbb{Q}) $ such that $ r \circ \iota = \text{id}_{\mathbb{Q}} $ for $ \iota: \mathbb{Q} \to B $ the specified map.

  By Baer's we must show that for every ideal of $ \mathbb{Z} $ and every map $ f: n\mathbb{Z} \to \mathbb{Q} $ we can extend it to a map $ \tilde{f}: \mathbb{Z} \to \mathbb{Q} $. This is easy enough. Suppose we have a map $ f: n\mathbb{Z} \to \mathbb{Q} $ and let $ \tilde{f}(1) = \frac{1}{n}f(n) $. Then, if $ i: n\mathbb{Z} \to \mathbb{Z} $ is the inclusion map, we have that $$ (\tilde{f}\circ i)(n) = \frac{1}{n}f(n \cdot n) = \frac{n}{n} f(n) = f(n). $$
  Hence $ \mathbb{Q} $ is injective and there exists a retraction $ r: B \to \mathbb{Q} $ so that $ r \circ \iota = \text{id}_{\mathbb{Q}} $. The splitting lemma then concludes the proof.
\end{exercise}

\begin{exercise}{6}
  For each prime $ p $ there is an embedding $ \mathbb{Z} [\frac{1}{p}]  / \mathbb{Z} \to \mathbb{Q}/\mathbb{Z} $. These glue together to form a map $ \alpha: \bigoplus_{p:\text{prime}} \mathbb{Z}  [\frac{1}{p}] / \mathbb{Z} \to \mathbb{Q}/\mathbb{Z} $. We just have to show that this map is an isomorphism.

  \textbf{Surjectivity}: Notice that for any rational number $ \frac{m}{pq} $ with $ p $ and $ q $ relatively prime there exists $ a,b $ such that $ \frac{m}{pq} = \frac{a}{p} + \frac{b}{q} $. Thus, for any rational number $ \frac{m}{\prod_{p} p^{n_p}} $ with $ n_p = 0 $ for all but finitely many $ p $'s we can write
  \begin{equation*}
   Q \coloneqq \frac{m}{\prod_{p} p^{n_p}} = \sum_{p} \frac{a_p}{p^{n_p}}
  .\end{equation*}
  Then $ Q \in \mathbb{Q}/\mathbb{Z} $ can be written as $ Q = \alpha\left(\oplus_{p}(a_p/p^{n_p})\right) $ showing that $ \alpha $ is surjective.

  \textbf{Injectivity}: Take $ \{a_p\} $ and $ \{k_p\} $ with all but finitely many of the $ a_p $ equal to zero such that $ \alpha \left( \oplus_{p} a_{p}/p^{k_p} \right) = 0 $. This tells us that $ \sum_{p} a_{p}/p^{k_p} = 0 $ which turns into
  \begin{equation*}
  \sum_{p} a_p \prod_{q \neq p} q^{k_q} = 0
  .\end{equation*}
  From this it follows that $ p^{k_p} | a_p $ for all $ p $ and so we must have that $ \left( \oplus_{p} a_{p}/p^{k_p} \right) $ is zero in $ \bigoplus_{p:\text{prime}} \mathbb{Z}  [\frac{1}{p}] / \mathbb{Z} $.

  We thus see that $ \alpha $ is an isomorphism.
\end{exercise}

\begin{exercise}{7}
  Let $ (A)_{i \geq 1} $ be a strictly increasing chain of submodules such that their union is equal to $ A $. Set $ M = A $ and $ N_i = A/A_i $. Then there is no map in no element in $ \bigoplus_{i = 1}^{\infty} \text{Hom}(A, A/A_i) $ which can get sent to the projection map $ p: A \to \bigoplus_{i = 1}^{\infty} A/A_i $. This is because we can only big a finite amount of non-zero maps in $ \bigoplus_{i = 1}^{\infty} \text{Hom}(A, A/A_i) $. Thus, if we pick any map $ \alpha \in  \bigoplus_{i = 1}^{\infty} \text{Hom}(A, A/A_i)  $ there is some index $ j $ such that if $ i > j $ then $ \alpha_i = 0 $. We can then pick some $ a \in A - A_j $ and have that $ \alpha_i(a) = 0 $ while $ p(a)_{j+1} \neq \alpha_{j+1}(a) $. Hence there is no isomorphism in general given that such an increasing sequence of submodules exists.

  This is certainly the case as you can take $ A = \bigoplus_{i = 1}^{\infty} \mathbb{Z} $ and $ A_{k} = \bigoplus_{i = 1}^{k} \mathbb{Z} $.
\end{exercise}
\end{document}
