\section{Abelian Categories} % (fold)
\label{sec:abelian_categories}
\begin{definition}
  Let $ \mathcal{C} $ be a category, and let $ x \in \mathcal{C} $. We say that $ x $ is \textbf{terminal} if for every $ c \in \mathcal{C} $, there is exactly one morphism $ c \to x $. Dually, we say that $ x $ is \textbf{initial} if for every $ c \in \mathcal{C} $, there is exactly one morphism $ x \to c $.
\end{definition}

\begin{definition}
   A \textbf{zero object} in a category is an object that is both initial and terminal.
\end{definition}

\subsection{Ab-enriched Categories} % (fold)
\label{sub:ab_enriched_categories}
\begin{definition}
   A \textbf{pre-additive} or \textbf{Ab-enriched} category is a category in which every hom-set is equipped with the structure of an abelian group, such that composition
   \begin{equation*}
     \text{Hom}(X,Y) \times \text{Hom}(Y, Z) \to \text{Hom}(X,Z)
   \end{equation*}
   is $ \mathbb{Z} $-bilinear.
\end{definition}

\begin{proposition}
   In an \textbf{Ab}-enriched category, any initial object is also terminal.
\end{proposition}
\begin{proof}
  Let $ * $ be initial. Then $ 1_{*} $ is the unique element of $ \text{Hom}(*, *) $, so $ 1_* $ is zero in this group. Then since composition respects the group structures, we have for any map $ f: A \to * $,
  \begin{equation*}
  f = 1_* \circ f = 0 \circ f = 0
  \end{equation*}
  so $ * $ is terminal.
\end{proof}

\begin{proposition}
  If $ \mathcal{C} $ is an \textbf{Ab}-enriched category, then so is its opposite category $ \mathcal{C}^{\text{op}} $.
\end{proposition}
\begin{proof}
  For $ X,Y \in \mathcal{C}^{\text{op}} $, the sets
  \begin{equation*}
    \text{Hom}_{\mathcal{C}^{\text{op}}}(X,Y) = \text{Hom}_{\mathcal{C}}(Y,X)
  \end{equation*}
  are already endowed with the structure of an abelian group. Thus, we only have to prove that composition is bilinear. Let $ X,Y,Z \in \mathcal{C} $ and let
  \begin{equation*}
    f,f' \in \text{Hom}_{\mathcal{C}^{\text{op}}}(X,Y),\quad g\in\text{Hom}_{\mathcal{C}^{\text{op}}}(Y,Z).
  \end{equation*}
  Then
  \begin{equation*}
    g\circ_{\text{op}}(f+f')= (f+f')\circ g = f\circ g + f'\circ g = g\circ_{\text{op}} f + g\circ_{\text{op}}f'.
  \end{equation*}
  Similarly, composition is linear in the other argument as well.
\end{proof}

\begin{proposition}
  In an \textbf{Ab}-enriched category $ \mathcal{C} $, a binary product is also a binary coproduct.
\end{proposition}
\begin{proof}
  Let $ X_1, X_2 $ be elments of an \textbf{Ab}-enriched category $ \mathcal{C} $. Suppose that $ X_1 $ and $ X_2 $ have a product $ X_1 \times X_2 $ in $ \mathcal{C} $, with projections $ p_k: X_1 \times X_2 \to X_k $. By definition of products, there are unique morphisms $ i_k:X_k \to X_1 \times X_2 $ such that the following diagrams commute.
\[\begin{tikzcd}
    {X_1} &&&&& {X_2} \\
    & {X_1 \times X_2} &&& {X_1\times X_2} \\
    {X_1} && {X_2} & {X_1} && {X_2}
    \arrow["{\text{id}}", from=1-1, to=3-1]
    \arrow["{i_1}", dashed, from=1-1, to=2-2]
    \arrow["0", bend left=45, from=1-1, to=3-3]
    \arrow["{p_2}"', from=2-2, to=3-3]
    \arrow["{p_1}", from=2-2, to=3-1]
    \arrow["{\text{id}}", from=1-6, to=3-6]
    \arrow["0"', bend right=45, from=1-6, to=3-4]
    \arrow["{i_2}"', dashed, from=1-6, to=2-5]
    \arrow["{p_1}", from=2-5, to=3-4]
    \arrow["{p_2}"', from=2-5, to=3-6]
\end{tikzcd}\]
  Then we have
  \begin{equation*}
  p_1\circ (i_1 p_1 + i_2p_2) = p_1, \quad p_2\circ(i_1p_1 + i_2p_2) = p_2.
  \end{equation*}
  By definition of products, $ \text{id} X_1 \times X_2 \times X_1 \times X_2 $ is the unique morphisms with $ p_k \circ \text{id}=p_k $ for each $ k $, so $ i_1p_1 + i_2p_2 = \text{id}_{X_1 \times X_2} $. We claim that
  \[\begin{tikzcd}
	{X_1} && {X_2} \\
	& {X_1\times X_2}
	\arrow["{i_1}", from=1-1, to=2-2]
	\arrow["{i_2}"', from=1-3, to=2-2]
\end{tikzcd}\]

is a universal cocone, so that $ X_1 \times X_2 = X_1 \coprod X_2 $. Suppose that
\[\begin{tikzcd}
	{X_1} && {X_2} \\
	& A
	\arrow["{f_1}"', from=1-1, to=2-2]
	\arrow["{f_2}", from=1-3, to=2-2]
\end{tikzcd}\]
is another cocone. Then we have a map
\begin{equation*}
\phi = f_1 \circ p_1 + f_2\circ p_2: X_1 \times X_2 \to A,
\end{equation*}
which is easily seen to give a commutative diagram
\[\begin{tikzcd}
	{X_1} && {X_2} \\
	& {X_1 \times X_2} \\
	& A.
	\arrow["{i_1}", from=1-1, to=2-2]
	\arrow["{i_2}"', from=1-3, to=2-2]
	\arrow["\phi", dashed, from=2-2, to=3-2]
	\arrow["{f_1}"', from=1-1, to=3-2]
	\arrow["{f_2}", from=1-3, to=3-2]
\end{tikzcd}\]
It remains to show that $ \phi $ is unique. To see this, note that for any such $ \phi $ we have
\begin{align*}
  \phi &= \phi\circ \text{id}_{X_1 \times X_2} \\
       &= \phi \circ (i_1p_1 + i_2 p_2) \\
       &= \phi i_1 \circ p_1 + \phi i_2\circ p_2 \\
       &= f_1\circ p_1 + f_2 \circ p_2
.\end{align*}
\end{proof}

\begin{proposition}
   In an \textbf{Ab}-enriched category, all binary coproducts are also binary products.
\end{proposition}
\begin{proof}
  This is dual to the previous proposition.
\end{proof}

\begin{definition}
  Let $ \mathcal{C} $ be an \textbf{Ab}-enriched category, and let $ x,y \in \mathcal{C} $. If $ x $ and $ y $ have a product in $ \mathcal{C} $, then it is called the biproduct of $ x $ and $ y $, which we denote by $ x \oplus y $.
\end{definition}

\begin{definition}
  Let $ F: \mathcal{A} \to \mathcal{B} $ be a functor between \textbf{Ab}-enriched categories. Then $ F $ is said to be \textbf{additive} if it preserves finite biproducts.
\end{definition}

\begin{lemma}
  For any ring $ R $, the category $ R $-\textbf{mod} is \textbf{Ab}-enriched.
\end{lemma}
% subsection Ab-enriched Categories (end)

\subsection{Additive Categories} % (fold)
\label{sub:additive_categories}
\begin{definition}
A category is \textbf{additive} if it is \textbf{Ab}-enriched and admits finite coproducts.
\end{definition}

\begin{lemma}
  Let $ \mathcal{A} $ be an additive category. Suppose that $ i: a \to b $ is a monomorphism in $ \mathcal{A} $ and $ \alpha \in \text{Hom}(a,b) $ is the zero morphism. Then $ a = 0 $.
\end{lemma}
\begin{proof}
  Let $ x \in \mathcal{A} $. Since $ \text{Hom}(a, x) $ is an abelian group, it contains at least one morphism (zero). Let $ f: a \to x $ be any morphism. Then
  \begin{equation*}
  \alpha\circ 0 = 0 = \alpha \circ f
  .\end{equation*}
  Since $ \alpha $ is a monomorphism, we have $ f = 0 $. Therefore $ a $ is initial, hence it is the zero object.
\end{proof}

\begin{lemma}
  Let $ \mathcal{A} $ be an additive category. Suppose that $ q: a \to b $ is an epimorphism in $ \mathcal{A} $. If $ q = 0 $, then $ b = 0 $.
\end{lemma}
\begin{proof}
  Since $ \mathcal{A} $ is additive, the opposite category $ \mathcal{A}^{\text{op}} $ is too. The map $ q $ is a monomorphism $ q: b \to a $ in $ \mathcal{A}^{\text{op}} $, and it is still the zero morphism. By the previous lemma we must therefore have that $ b $ is the zero object in $ \mathcal{A}^{\text{op}} $, hence in $ \mathcal{A} $.
\end{proof}

\begin{lemma}
   For any ring $ R $, the category $ R $-\textbf{mod} is additive.
\end{lemma}
\begin{proof}
  We know that the direct sum exists and is a coproduct in $ R $-\text{mod}.
\end{proof}
% subsection Additive Categories (end)

\subsection{Pre-abelian Categories} % (fold)
\label{sub:pre_abelian_categories}
\begin{definition}
   An additive category is \textbf{pre-abelian} if every morphism has a kernel and cokernel.
\end{definition}

\begin{lemma}
  Let $ \mathcal{A} $ be a pre-abelian category. Every monomorphism has kernel $ 0 $, and every epimorphism has cokernel $ 0 $.
\end{lemma}

\begin{proof}
  Let $ i: a \to b $ be a monomorphism in $ \mathcal{A} $. Let
  \begin{equation*}
    \text{Ker}i \xrightarrow{\text{ker}i} a
  \end{equation*}
  be the kernel of $ i $. Then $ i \circ \text{ker}i = 0 = i\circ 0 $, so $ \text{ker}i $ is the zero morphism (since $ i $ is a monomorphism). Since $ \text{ker}i $ is monomorphism, we have $ \text{Ker}i = 0 $.
\end{proof}

\begin{lemma}
  For any ring $ R $, the category $ R $-\textbf{mod} is pre-abelian.
\end{lemma}
% subsection Pre-abelian Categories (end)

\subsection{Abelian Categories} % (fold)
\label{sub:abelian_categories}
\begin{definition}
   A pre-abelian category is \textbf{abelian} if every monomorphism is the kernel of its cokernel and every epimorphism is the cokernel of its kernel.
\end{definition}

\begin{lemma}
   The category of left $ R $-modules is an abelian category.
\end{lemma}

\begin{proof}
  Let $ i: A\to B $ be a monomorphism of $ R $-modules. Then $ \text{Coker}i = B/i(A) $ and the cokernel map is the quotient $ q: B \to B/i(A) $ with $ q(b) = b + i(A) $. It is clear that $ i(A) $ = $ \text{Ker}q $ in the set-theoretic sense, so $ i $ exhibits $ A $ as the kernel of $ q $.

  Let $ q: A \to B $ be an epimorphism of $ R $-modules. Let $ i: \text{Ker} q \to A $ be the inclusion. Then $ \text{Coker}i = A/\text{Ker}q \cong B $, so $ q $ exhibits $ B $ as the cokernel of $ i $.
\end{proof}

\begin{lemma}
  If $ \mathcal{A} $ is abelian, then so is $ \mathcal{A}^{\text{op}} $.
\end{lemma}

\begin{proof}
  Duality.
\end{proof}

\begin{lemma}
  If $ \mathcal{A} $ is an abelian category and $ \mathcal{C} $ is any category, then $ \text{Fun}(\mathcal{C}, \mathcal{A}) $ is abelian.
\end{lemma}
% subsection Abelian Categories (end)

\subsection{Connection with $ R $-mod} % (fold)
\label{sub:connection_with_r_}

\begin{theorem}[Freyd-Mitchell Embedding Theorem]
  Let $ \mathcal{A} $ be a small abelian category. Then there is a ring $ R $ and an exact, fully faithful functor $ F:\mathcal{A} \to R\mathbf{mod} $. This functor embeds $ \mathcal{A} $ as a full subcategory in $ R $-\textbf{mod}, by which we mean that for all $ M,N \in \mathcal{A} $, we have
  \begin{equation*}
    \text{Hom}_{\mathcal{A}}(M,N) \cong \text{Hom}_R(F(M),F(N)).s
  .\end{equation*}
\end{theorem}

\begin{lemma}
  The Freyd-Mitchell embedding preserves kernels and cokernels.
\end{lemma}
\begin{proof}
  Let $ f: x \to y $ be a morphism in an abelian category $ \mathcal{A} $, and let $ F: \mathcal{A}\to R\mathbf{mod} $ be the Freyd-Mitchell embedding. Consider the sequence
  \begin{equation*}
    0 \to \text{Ker}f \xrightarrow{i} x \xrightarrow{f} y \xrightarrow{q} \text{Coker}f \to 0
  .\end{equation*}
\end{proof}

\begin{lemma}
  Let $ \mathcal{A} $ be an abelian category and let $ F:\mathcal{A} \to R$-\textbf{mod} be the embedding from before. Then $ F(0)=0 $.
\end{lemma}
% subsection Connection with $ R $ (end)
% section Abelian Categories (end)
