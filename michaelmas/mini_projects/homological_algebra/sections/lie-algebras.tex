\section{Lie Algebras and $ \mathfrak{g} $-modules} % (fold)
\label{sec:Lie Algebras}
\subsection{Basics of Lie Algebras} % (fold)
\label{sub:Basics of Lie Algebras}
We will assume throughout this project that $ k $ is a field.
\begin{definition}[Lie Algebra]
  A \textbf{Lie algebra} is a $ k $-module $ \mathfrak{g} $ equipped with a bilinear skew-symmetric map $ [-,-]: \mathfrak{g} \times \mathfrak{g} \to \mathfrak{g} $ which satisfies the Jacobi identity:
  \begin{equation}
    \forall x,y,z \in \mathfrak{g}\, [x,[y,z]] + [z, [x, y]] + [y, [z, x]] = 0
    \label{eq:jacobi}
  .\end{equation}
  A morphism of Lie algebras $ \phi: \mathfrak{g} \to \mathfrak{h} $ is a $ k $-linear map which respects the bracket, i.e.,
  \begin{equation}
    \phi([x,y]) = [\phi(x), \phi(y)].
    \label{eq:liehom}
  \end{equation}
\end{definition}

For any associative algebra $ A $ there is a natural Lie algebra $ \mathfrak{a}\coloneqq \text{Lie}(A) $ where, for $ x,y \in A $, one defines
\begin{equation*}
  [x,y] = xy - yx
.\end{equation*}
It is straightforward to verify that $ [-,-]: \mathfrak{a} \times \mathfrak{a} \to \mathfrak{a} $ satisfies the Jacobi identity.

\begin{lemma}
  \label{lem:functoriality}
  The assignment $ \text{Lie}:k\text{-}\mathbf{Alg} \to k\text{-}\mathbf{LieAlg} $ as defined above is a functor.
\end{lemma}
\begin{proof}
  Let $ f: A \to B $ be a morphism of $ k $-algebras. The map $ \text{Lie}(f):\text{Lie}(A) \to \text{Lie}(B) $ defined for $ x \in \text{Lie}(A) $ by $ (\text{Lie}(f))(x)=f(x) $ is a Lie algebra morphism. To see this, let $ x,y \in A $, then
  \begin{align*}
    (\text{Lie}(f))([x,y]) &= f([x,y]) \\
                           &= f(xy - yx) \\
                           &= f(xy) - f(yx) \\
                           &= f(x)f(y) - f(y)f(x) \\
                           &= [f(x), f(y)] \\
                           &= [(\text{Lie}(f))(x), (\text{Lie}(f))(y)]
  \end{align*}
  and hence $ \text{Lie}(f) $ is indeed a Lie algebra homomorphism. It trivially holds that $ \text{Lie}(1_A) = 1_{\text{Lie}(A)} $. Hence the only thing that remains to be checked is that composition is preserved. Let $ f \in \text{Hom}_{k\text{-}\mathbf{Alg}}(A,B) $, $ g \in \text{Hom}_{k\text{-}\mathbf{Alg}}(B,C) $. We then have that
  \begin{align*}
    \text{Lie}(g \circ f) &= g \circ f \\
                          &= \text{Lie}(g) \circ \text{Lie}(f)
  .\end{align*}
\end{proof}

It turns out that Lie has a left adjoint $ U: k\text{-}\mathbf{LieAlg} \to k\text{-}\mathbf{Alg} $ called the universal enveloping algebra. In order to define this functor we must first talk about tensor algebras.

\begin{definition}[Tensor algebra]
  If $ M $ is any $ k $-module, then the \textbf{tensor algebra} $ T(M) $ is the graded associative $ k $-algebra with unit generated by $ M $:
  \begin{equation}
    T(M) \coloneqq \bigoplus_{n = 0}^{\infty} M^{\otimes n}
    \label{eq:gaa}
  \end{equation}
  where $ M^{\otimes n} = \bigotimes_{i = 1}^{n} M $ for $ n>0 $ and $ M^{\otimes 0}=k $. Multiplication is defined by concatenation of terms in the obvious way.
\end{definition}

It is straightforward to verify that $ T $ defines a functor from the category of $ k $-modules to the category of associative unital $ k $-algebras. Using $ T $ we can the proceed to define the universal enveloping algebra of a Lie algebra $ \mathfrak{g} $:

\begin{definition}[Universal enveloping algebra]
  If $ \mathfrak{g} \in k\text{-}\mathbf{LieAlg} $, then the \textbf{universal enveloping algebra} $ U\mathfrak{g} $ is the quotient of $ T\mathfrak{g} $ by the two sided ideal $ I_\mathfrak{g} $ generated by the relations
  \begin{equation}
    i(x) \otimes i(y) - i(y) \otimes i(x) - i([x,y]),\quad (x,y \in \mathfrak{g}).
    \label{eq:relations}
  \end{equation}
  where $ i: \mathfrak{g} \to T\mathfrak{g} $ is the inclusion map.
\end{definition}

The relations in Equation~\ref{eq:relations} makes $ i:\mathfrak{g} \to \text{Lie}(U\mathfrak{g}) $ into a Lie algebra homomorphism. Technically $ i $ is the inclusion $ i: \mathfrak{g} \to T\mathfrak{g} $ followed by the projection $ p: T\mathfrak{g} \to U\mathfrak{g} $ but we will denote it by $ i $ nonetheless (this is justified as we shall see).


Let $ \{e_{\alpha}\} $ be a fixed and ordered basis for some Lie algebra $ \mathfrak{g} $ over $ k $.
If $ I = (\alpha_1, \ldots, \alpha_p) $ is a sequence of indices we will abuse notation to let $ e_I $ denote the image of $ e_{\alpha_1} \otimes \cdots \otimes e_{\alpha_p} $ in $ U\mathfrak{g} $. The sequence $ I $ is called \textbf{increasing} if $ \alpha_1 \leq \ldots \leq \alpha_p $. The empty sequence $ \phi = () $ is also regarded as increasing and we set $ e_\phi = 1 $. Note that if $ I = (\alpha) $ is a single index then $ e_\alpha \in \mathfrak{g} $ while $ e_{(\alpha)} \in U\mathfrak{g} $.

\begin{theorem}[Poincar\'e-Birkhoff-Witt Theorem]
  \label{thm:pbw}
  If $ \mathfrak{g} $ is a free $ k $-module, then $ U\mathfrak{g} $ is also a free $ k $-module. If $ \{e_{\alpha}\} $ is an ordered basis of $ \mathfrak{g} $, then the elements $ e_I $ with $ I $ an increasing sequence form a basis of $ U\mathfrak{g} $.
\end{theorem}
\begin{proof}
  See \cite[V.2]{jacobson1979lie}.
\end{proof}

Note that since we assume $ k $ is a field it follows that every $ k $-module is free and hence the Poincar\'e-Birkhoff-Witt theorem will always apply for us.

\begin{corollary}
  The map $ i: \mathfrak{g} \to U\mathfrak{g} $ as defined earlier is an injection identifying $ \mathfrak{g} $ with $ i(\mathfrak{g}) $. Moreover $ U\mathfrak{g} $ is the free $ k $-algebra generated by $ i(\mathfrak{g}) $.
\end{corollary}
\begin{proof}
  As each basis element of $ \mathfrak{g} $ is mapped to a basis element of $ U\mathfrak{g} $ the first part follows from the Poincar\'e-Birkhoff-Witt theorem. The second part also follows from the Poincar\'e-Birkhoff-Witt theorem: if $ \{e_{\alpha}\} $ is an ordered basis for $ \mathfrak{g} $ then $ i(e_\alpha) = e_{(\alpha)} $ and since $ e_{(\alpha)}e_{(\beta)} = e_{(\alpha, \beta)} $ it follows that all basis elements of $ U\mathfrak{g} $ can be generated from the multiplication of elements of the form $ e_{(\alpha)} $. As we define the multiplication of zero elements to be the unit $ 1 \in k $ we have that $ i(\mathfrak{g}) $ does indeed generate $ U\mathfrak{g} $ as a free $ k $-algebra.
\end{proof}

Later we will be interested in creating a projective resolution of $ U\mathfrak{g} $-modules and a good place to start is by defining a $ k $-algebra homomorphism $ U\mathfrak{g} \to k $.
\begin{definition}[Augmentation ideal]
  Let $ \{e_{\alpha}\} $ be an ordered basis for $ \mathfrak{g} $. The \textbf{augmentation ideal} of $ \mathfrak{g} $ is defined to be the kernel of the $ k $-algebra homomorphism $ \epsilon: U\mathfrak{g} \to k $ where $ \epsilon(e_{(\alpha)}) = 0 $ for all $ \alpha $. We denote this ideal by $ \mathfrak{J} = \text{ker}(\epsilon) $.
\end{definition}
It is fairly straightforward to see that $ k \cong U\mathfrak{g}/\mathfrak{J} $ as the only things which survive the quotienting process is the $ k $ in the 0th degree of $ U\mathfrak{g} $.

More can be said about the functor $ U $:
\begin{lemma}
  \label{lem:ufunct}
  The construction $ U:k\text{-}\mathbf{LieAlg} \to k\text{-}\mathbf{Alg} $ is functorial.
\end{lemma}
\begin{proof}
  We first note that for $ f \in \text{Hom}_{k\text{-}\mathbf{LieAlg}}(\mathfrak{g}, \mathfrak{h}) $, $ Uf: U\mathfrak{g} \to U\mathfrak{h} $ and $ x_1, \ldots, x_n \in \mathfrak{g} $ we have that
  \begin{equation}
    Uf(x_1\cdots x_n) = f(x_1)\cdots f(x_n)
    \label{eq:uf}
  \end{equation}
  where we abuse notation and let $ x_1\cdots x_n $ denote the equivalence class of $ x_1 \otimes \cdots \otimes x_n $ in $ U\mathfrak{g} $.
  This is well defined as for $ x,y \in \mathfrak{g} $ we have
  \begin{equation*}
    f(x \otimes y - y \otimes x - [x,y]) = f(x) \otimes f(y) - f(y) \otimes f(x) - [f(x), f(y)] \in I_{\mathfrak{h}}
  \end{equation*}
  where $ I_{\mathfrak{h}} $ denotes the two sided ideal used in the definition of $ U $. Hence $ Tf(I_\mathfrak{g}) \subset I_\mathfrak{h} $ and since $ Tf|_k = 1_k $ this descends to a morphism $ Uf:U\mathfrak{g} \to U\mathfrak{h} $.

  From Equation~\ref{eq:uf} it follows that
  \begin{equation*}
    U 1_{\mathfrak{g}} = 1_{U\mathfrak{g}}.
  \end{equation*}
  For $ x_1, \ldots, x_n \in \mathfrak{g} $, $ f \in \text{Hom}_{k\text{-}\mathbf{LieAlg}}(\mathfrak{g}, \mathfrak{h})  $, $  g \in \text{Hom}_{k\text{-}\mathbf{LieAlg}}(\mathfrak{h}, \mathfrak{l})  $ we have that
  \begin{align*}
    U(g \circ f)(x_1 \cdots x_n) &= g(f(x_1)) \cdots gf(x_n) \\
                                                     &= Ug(f(x_1) \cdots f(x_n)) \\
                                                     &= (Ug \circ Uf)(x_1 \cdots x_n)
  \end{align*}
  and hence
  \begin{equation}
    U(g \circ f) = Ug \circ Uf
  .\end{equation}
  We thus see that $ U:k\text{-}\mathbf{LieAlg} \to k\text{-}\mathbf{Alg} $ defines a functor.
\end{proof}

\begin{proposition}
  \label{prop:adjointu}
  The functor $  U:k\text{-}\mathbf{LieAlg} \to k\text{-}\mathbf{Alg}  $ is left adjoint to $ \text{Lie}:k\text{-}\mathbf{Alg} \to k\text{-}\mathbf{LieAlg} $.
\end{proposition}
\begin{proof}
  We do this by explicitly constructing the unit and counit. Define the unit $ \eta: 1_{k\text{-}\mathbf{LieAlg}} \implies \text{Lie}U $ componentwise by the inclusion map $ i_\mathfrak{g}: \mathfrak{g} \to \text{Lie}(U\mathfrak{g}) $
  i.e., $\eta_\mathfrak{g} = i_\mathfrak{g}$.
  We define the counit $ \epsilon: U\text{Lie} \implies 1_{k\text{-}\mathbf{Alg}} $ to be the evaluation map, i.e., for $ \{e_{\alpha}\} $ an ordered basis on $ \text{Lie}(A) $ and $ I = (\alpha_1, \ldots, \alpha_n) $ an increasing sequence we let
  \begin{equation}
    \epsilon_A(e_I) = (e_{\alpha_1}(\cdots (e_{\alpha_{n-1}}e_{\alpha_n})\cdots)).
  \end{equation}
  It is straightforward to verify that both $ \eta $ and $ \epsilon $ are natural constructions. To see this for $ \eta $ we let $ \phi:\mathfrak{g} \to \mathfrak{h} $ be a map between two $ k $-Lie algebras. We must then verify commutativity of the following diagram:
  \[\begin{tikzcd}
	  {\mathfrak{g}} & {\mathfrak{h}} \\
	  {\text{Lie}U\mathfrak{g}} & {\text{Lie}U\mathfrak{h}.}
	  \arrow["\phi", from=1-1, to=1-2]
	  \arrow["{\text{Lie}U\phi}"', from=2-1, to=2-2]
	  \arrow["{i_\mathfrak{g}}"', from=1-1, to=2-1]
	  \arrow["{i_\mathfrak{h}}", from=1-2, to=2-2]
  \end{tikzcd}\]
  It suffices to show commutativity on basis elements. Hence, let $ \{e_\alpha\} $ and $ \{d_\beta\} $ be ordered basis sets on $ \mathfrak{g} $ and $ \mathfrak{h} $ respectively. We then have that
  \begin{align*}
    i_\mathfrak{h}(\phi(e_\alpha)) &=i_\mathfrak{h}\left( \sum_{\beta} y_\beta d_\beta \right) \\
                                   &= \sum_{\beta} y_\beta d_{(\beta)} \\
                                   &= \text{Lie}U\phi \left( e_{(\alpha)} \right) \\
                                   &= \text{Lie}U\phi(i_\mathfrak{g}(e_\alpha))
  \end{align*}
  as desired. A similar argument works to show that $ \epsilon $ is natural.

  We must now verify the triangle identities:
  \begin{align*}
    \epsilon U \circ U\eta &= 1_U \\
    \text{Lie}\,\epsilon \circ \eta\,\text{Lie} &= 1_{\text{Lie}}.
  \end{align*}
  It suffices to verify it on basis elements. Let $ \{e_\alpha\} $ be an ordered basis on some $ k $-Lie algebra $ \mathfrak{g} $ and $ \{x_\beta\} $ an ordered basis on some associative $ k $-algebra $ A $ (both have basis as $ k $ is a field). Let $ I = (\alpha_1, \ldots, \alpha_n) $ be an increasing sequence. We then have
  \begin{align*}
    (\epsilon_{U\mathfrak{g}} \circ U\eta_\mathfrak{g})(e_I) &= \epsilon_{U\mathfrak{g}}(Ui_\mathfrak{g}(e_I)) \\
                                                             &= \epsilon_{U\mathfrak{g}}(\overline{e_I}) \\
                                                             &= e_I
  \end{align*}
  as desired, where $ \overline{e_I} $ is the equivalence class of $ e_I $ in $ U^2\mathfrak{g} $. Similarly, for $ x_\beta \in \{x_\beta\} $ we have
  \begin{align*}
    (\text{Lie}\,\epsilon_A \circ \eta_{\text{Lie}(A)})(x_\beta) &= \text{Lie}\,\epsilon_A (x_{(\beta)}) \\
                                                        &= x_\beta
  \end{align*}
  as desired. Hence $ \eta $ and $  \epsilon $ satisfy the requirement for $ U $ being left adjoint to $ \text{Lie} $.
\end{proof}


% subsection Basics of Lie Algebras (end)


\subsection{$\mathfrak{g}$-modules} % (fold)
\label{sub:g-modules}
Lie algebras are not rings and so the concept of $ \mathfrak{g} $-modules is not defined out of the box. However, there is a natural way in which a $ \mathfrak{g} $-module can make sense.
\begin{definition}[Left $ \mathfrak{g} $-module]
  Let $ \mathfrak{g} $ be a Lie algebra over the field $ k $. A \textbf{left} $ \mathfrak{g} $-\textbf{module} is a $ k $-module $ M $ together with a bilinear action map $ \mathfrak{g} \otimes_k M \to M $ such that $ x \otimes m $ maps to $ xm $ and
  \begin{equation}
    [x,y]m = x(ym) - y(xm).
    \label{eq:whoknow}
  \end{equation}
  Morphisms of left $ \mathfrak{g} $-modules $ M $ and $ N $ are $ k $-linear maps
  $ f: M \to N $ such that the following diagram commutes
  \[\begin{tikzcd}
	  {\mathfrak{g} \otimes_k M} & {\mathfrak{g} \otimes_k N} \\
	  M & N
	  \arrow[from=1-1, to=2-1]
	  \arrow["f"', from=2-1, to=2-2]
	  \arrow[from=1-2, to=2-2]
	  \arrow["{\text{1}_\mathfrak{g} \otimes f}", from=1-1, to=1-2]
  \end{tikzcd}\]
  where the vertical maps are the action maps.
\end{definition}

In order to use (co)homological methods on the category of $ \mathfrak{g} $-modules we need that it is abelian.
\begin{lemma}
  The category $ \mathfrak{g} $-\textbf{Mod} is preadditive.
\end{lemma}
\begin{proof}
  First, note that $ \text{Hom}_{\mathfrak{g}\text{-}\mathbf{Mod}}(M, N) $ is a subset of $ \text{Hom}_{k\text{-}\mathbf{Mod}}(M, N) $. In fact, it is a submodule since for $ \alpha \in k $ and $ f,g \in  \text{Hom}_{\mathfrak{g}\text{-}\mathbf{Mod}}(M, N) $ we have $ \alpha f \in  \text{Hom}_{\mathfrak{g}\text{-}\mathbf{Mod}}(M, N)  $ and $ f + g \in  \text{Hom}_{\mathfrak{g}\text{-}\mathbf{Mod}}(M, N)  $. Hence $  \text{Hom}_{\mathfrak{g}\text{-}\mathbf{Mod}}(M, N)  $ is a $ k $-submodule of $  \text{Hom}_{\mathfrak{k}\text{-}\mathbf{Mod}}(M, N)  $ and so is certainly abelian. That the composition $ \mathbb{Z} $-bilinear follows immediately since $  \text{Hom}_{\mathfrak{g}\text{-}\mathbf{Mod}}(M, N)  $ is a $ k $-submodule of $  \text{Hom}_{\mathfrak{k}\text{-}\mathbf{Mod}}(M, N)  $. Hence $ \mathfrak{g} $-\textbf{Mod} is preadditive.
\end{proof}

\begin{lemma}
  The category $ \mathfrak{g} $-\textbf{Mod} is additive.
\end{lemma}
\begin{proof}
  It is enough to show that $ \mathfrak{g} $-\textbf{Mod} has a zero object and binary coproducts. The zero object $ 0 \in k $-\textbf{Mod} works as a zero object in $ \mathfrak{g} $-\textbf{Mod} as well. We thus only need to show that it has binary coproducts.

  To this end, let $ M,N \in \mathfrak{g} $-\textbf{Mod}. Consider the direct sum of $ M $ and $ N $ as $ k $-modules $ M \oplus N $. We will give this a $ \mathfrak{g} $-module structure and show that satisfies the property of being the binary coproduct of $ M $ and $ N $ in the category of $ \mathfrak{g} $-modules. Note that $ \mathfrak{g} \otimes_k (M \oplus N) \cong (\mathfrak{g} \otimes_k M) \oplus (\mathfrak{g} \otimes_k N) $ as $ k $-modules. The two maps $ \mathfrak{g} \otimes_k M \to M $ and $ \mathfrak{g} \otimes_k N \to N $ then combine to give a map
  \begin{equation}
    \mathfrak{g} \otimes_k (M \oplus N) \cong (\mathfrak{g} \otimes_k M) \oplus (\mathfrak{g} \otimes_k N)  \to M \oplus N.
    \label{eq:bincop}
  \end{equation}
  For $ x,y \in \mathfrak{g} $ and $ m + n \in M \oplus N $ we then also have that
  \begin{align*}
    [x,y](m + n) &= [x,y]m + [x,y]n \\
                 &= x(ym) - y(xm) + x(yn) - y(xn) \\
                 &= x(y(m + n)) - y(x(m + n)),
  \end{align*}
  which is a necessary criteria for $ M \oplus N $ being given a $ \mathfrak{g} $-module structure. The inclusion maps $ M \to M \oplus N $ and $ N \to M \oplus N $ are then also $ \mathfrak{g} $-module morphisms.

  The only remaining thing to show then is that $ M \oplus N $ satisfies the universality property. Thus, let $ f: M \to P $ and $ g: N \to P $ be two $ \mathfrak{g} $-module morphisms. Viewing them as $ k $-module morphisms we know there is a $ k $-module morphism $ f \oplus g: M \oplus N \to P $. What remains to show is that it also a $ \mathfrak{g} $-module morphism. In other words, that the following diagram commutes
  \[\begin{tikzcd}
	  {\mathfrak{g} \otimes_k (M \oplus N)} & {\mathfrak{g} \otimes_k P} \\
	  {M\oplus N} & P
	  \arrow[from=1-1, to=2-1]
	  \arrow["{f\oplus g}"', from=2-1, to=2-2]
	  \arrow[from=1-2, to=2-2]
	  \arrow["{1_\mathfrak{g} \otimes (f \oplus g)}", from=1-1, to=1-2]
  \end{tikzcd}\]
  Pointwise we have that for $ m \in M $, $ n \in N $, and $ x \in \mathfrak{g} $
  \begin{align*}
    (f \oplus g) (x(m+n)) &= (f \oplus g)(xm + xn) \\
                          &= f(xm) + g(xn) \\
                          &= xf(m) + xg(n) \\
                          &= x(f(m) + g(n)) \\
                          &= x((f \oplus g)(m + n))
  \end{align*}
  and hence the diagram does indeed commute which means that $ f \oplus g $ is a $ \mathfrak{g} $-module morphism. Thus $ M \oplus N $ with the imposed $ \mathfrak{g} $-module structure satisfies the universality property and so $ \mathfrak{g} $-\textbf{Mod} has finite coproducts.
\end{proof}

\begin{lemma}
  The category $ \mathfrak{g} $-\textbf{Mod} is preabelian.
\end{lemma}
\begin{proof}
  This follows if we can show that for a given $ f \in \text{Hom}_{\mathfrak{g}\text{-}\mathbf{Mod}}(M ,N) $ the $ k $-modules $ \text{ker}(f) $ and $ \text{coker}(f) $ are also the kernel and cokernel, respectively, of $ f $.

  Suppose $ m \in \text{ker}(f) $ and $ x \in \mathfrak{g} $. Then $ f(xm) = xf(m) = 0 $ and hence $ \mathfrak{g} \otimes_k M \to M $ restricts to a map $ \mathfrak{g} \otimes_k \text{ker}(f) \to \text{ker}(f) $ which makes the following diagram commute:
  \[\begin{tikzcd}
	  {\mathfrak{g} \otimes_k \text{ker}(f)} & {\mathfrak{g} \otimes_k M} \\
	  {\text{ker}(f)} & M
	  \arrow[from=1-1, to=2-1]
	  \arrow["i"', from=2-1, to=2-2]
	  \arrow[from=1-2, to=2-2]
	  \arrow["{1_\mathfrak{g} \otimes i}", from=1-1, to=1-2]
  \end{tikzcd}\]
  Suppose next that there is a morphism $ g \in \text{Hom}_{\mathfrak{g}\text{-}\mathbf{Mod}}(P, M) $ such that $ g\circ f = 0 $. There is then a unique $ k $-module morphism $ \alpha:P \to \text{ker}(f) $ such that $ g = i\circ \alpha $. We just need to show that $ \alpha $ is also a $ \mathfrak{g} $-module morphism. Let $ x \in \mathfrak{g} $ and $ p \in P $, we then have
  \begin{align*}
    i(\alpha(xp)) &= g(xp) \\
                  &= xg(p) \\
                  &= x i(\alpha(p)) \\
                  &= i(x\alpha(p))
  \end{align*}
  and since $ i $ is injective we have $ \alpha(xp) = x\alpha(p) $ showing that $ \alpha $ is a $ \mathfrak{g} $-module morphism. Hence the universality of $ \text{ker}(f) $ has been established. That $ [x,y]m = x(ym) - y(xm) $ for $ m \in \text{ker}(f) $ follows immediately since $ \text{ker}(f) $ is a submodule of $ M $. Hence $ \text{ker}(f) $ is indeed the kernel of $ f $.

  Now, suppose $ n \in \text{im}(f) $ and $ x \in \mathfrak{g} $. Then $ n = f(m) $ for some $ m \in M $. Hence $ xn = xf(m) = f(xm) $ which means $ xn \in \text{im}(f) $ and so we have a commutative diagram
  \[\begin{tikzcd}
	  {\mathfrak{g} \otimes_k N} & {\mathfrak{g} \otimes_k \text{coker}(f)} \\
	  N & {\text{coker}(f)}
	  \arrow[from=1-1, to=2-1]
	  \arrow["\pi"', from=2-1, to=2-2]
	  \arrow["{1 \otimes \pi}", from=1-1, to=1-2]
	  \arrow[from=1-2, to=2-2]
  \end{tikzcd}\]
  where $ \pi $ is the projection. For $ \overline{n} \in \text{coker}(f) $ and $ x,y \in \mathfrak{g} $ the equality $ [x,y]\overline{n} = x(y\overline{n}) - y(x\overline{n}) $ follows directly from the commutativity of the above diagram. Hence, what remains is to check the universality condition. Let therefore $ g \in \text{Hom}_{\mathfrak{g}\text{-}\mathbf{Mod}}(N, P) $ such that $ g\circ f = 0 $. By the universality of $ \text{coker}(f) $, as a $ k $-module, there exists a unique $ k $-module map $ \alpha: P \to \text{coker}(f) $ such that $ g = \alpha \circ \pi $. We need to show that $ \alpha $ is a $ \mathfrak{g} $-module map as well. Hence let $ [n] \in \text{coker}(f) $ and $ x \in \mathfrak{g} $. We then have that
  \begin{align*}
    \alpha(x[n]) &= \alpha(x\pi(n)) \\
                 &= \alpha(\pi(xn)) \\
                 &= x\alpha(\pi(n)) \\
                 &= x\alpha([n])
  \end{align*}
  as desired.
\end{proof}

\begin{proposition}
  \label{prop:abelian}
  The category $ \mathfrak{g} $-\textbf{Mod} is abelian.
\end{proposition}
\begin{proof}
  What remains to show is that every monomorphism is the kernel of some morphism and every epimorphism is the cokernel of some morphism.

  Let therefore $ f \in \text{Hom}_{\mathfrak{g}\text{-}\mathbf{Mod}}(M, N) $ be a monomorphism. We then have that $ f $ is the kernel (considered as a $ k $-module) of $ \mathfrak{g} $-module morphism $ \pi: N \to \text{coker}(f) $. As $ k $-\textbf{Mod} is abelian, we have $ \text{ker}(\pi) = \text{im}(f) \cong M $. The proof of the previous lemma then gives the result.

  The same argumentation also works for epimorphisms. We therefore have that $ \mathfrak{g} $-\textbf{Mod} is abelian.
\end{proof}

More can be said about the structure of $ \mathfrak{g} $-\textbf{Mod}:
\begin{proposition}
  \label{prop:monoidal}
  The category $ \mathfrak{g} $-\textbf{Mod} is closed symmetric monoidal where the action map on $ M \otimes_k N $ is given by $ x(m \otimes n) = xm\otimes n + m\otimes xn $.
\end{proposition}
\begin{proof}
  See Appendix~\ref{sec:Closed Symmetric Monoidal Structure on}.
\end{proof}

Now, as $ \mathfrak{g} $-\textbf{Mod} is abelian, it makes sense to talk about projective and injective resolutions for objects $ M \in \mathfrak{g} $-\textbf{Mod}. We can then use these resolutions to talk about Ext and Tor which we use to define the (co)homology of $ M $.

In order to be able to do this we need that every $ \mathfrak{g} $-module has a projective and injective resolution.
\begin{lemma}
  \label{lem:eqcat}
  If $ F:\mathcal{A} \rightleftarrows \mathcal{B}: G $ is an equivalence of abelian categories then $ \mathcal{A} $ has enough projectives/injectives if and only if $ \mathcal{B} $ has enough projectives/injectives.
\end{lemma}
\begin{proof}
  Let $ F:\mathcal{A} \rightleftarrows \mathcal{B}:G $ be an equivalence of categories. Then $ F $ and $ G $ preserves projective/injective objects. Without loss of generality, assume that $ \mathcal{B} $ has enough projectives and let $ M \in \mathcal{A} $. We want to find a projective object $ P \in \mathcal{A} $ and an epimorphism $ \epsilon: P \to M $. Since $ \mathcal{B} $ has enough projectives we know that there exists a projective object $ P' \in \mathcal{B} $ and an epimorphism $ \epsilon': P' \to F(M) $. We then have that $ G(P') $ is projective in $ \mathcal{A} $ and $ G(\epsilon'):G(P') \to GF(M) $ is an epimorphism. We then use the isomorphism $ \eta_M^{-1}:GF(M) \cong M $ and define $ \epsilon = \eta_M^{-1} \circ \epsilon' $ which is still an epimorphism. Letting $ P = G(P') $ we have an epimorphism $ \epsilon: P \to M $ from a projective object $ P $ to $ M $ showing that $ \mathcal{A} $ has enough projectives.

  The other direction is proved similarly and the case for enough injectives follows by duality.
\end{proof}

\begin{lemma}
  \label{lem:ex7.2.2}
  Let $ E = \text{End}_k(M) $ be the associative $ k $-algebra of $ k $-module endomorphisms of a $ k $-module $ M $. For $ \mathfrak{g} $ a Lie algebra over $ k $ we have the collection of maps $ \mathfrak{g} \otimes_k M \to M $ making $ M $ into a $ \mathfrak{g} $-module is in a natural bijection with Lie algebra homomorphisms $ \mathfrak{g} \to \text{Lie}(E) $ (where naturality is considered with respect to $ \mathfrak{g} $). Thus a $ \mathfrak{g} $-module may equivalently be described as a $ k $-module $ M $ together with a Lie algebra homomorphism $ \mathfrak{g} \to \text{Lie}(\text{End}_k(M)) $.
\end{lemma}
\begin{proof}
  This is Exercise 7.2.2 in \cite{weibel1994homological}.

  Let
  \begin{equation}
    \mathcal{M}_\mathfrak{g} = \{f: \mathfrak{g} \otimes_k M \to M \mid f \text{ makes } M \text{ into a } \mathfrak{g}\text{-module}\}
  .\end{equation}
  We want to create a bijection
  \begin{equation*}
    \Phi: \mathcal{M}_{\mathfrak{g}} \rightleftarrows \text{Hom}_{k\text{-}\mathbf{LieAlg}}(\mathfrak{g}, \text{Lie}(E)): \Psi
  .\end{equation*}
  Let $ f \in \mathcal{M}_{\mathfrak{g}} $, $ x \in \mathfrak{g} $, $ m \in M $, and define $ \Phi(f) $ pointwise by
  \begin{equation}
    \Phi(f)(x)(m) = f(x \otimes m) = xm.
  \end{equation}
  To see that this is a Lie algebra homomorphism, let $ x, y \in \mathfrak{g} $, then
  \begin{align*}
    \Phi(f)([x,y])(m) &= f([x,y] \otimes m) \\
                      &= x(ym) - y(xm) \\
                      &= \Phi(f)(x)(\Phi(f)(y)(m)) - \Phi(f)(y)(\Phi(f)(x)(m)) \\
                      &= (\Phi(f)(x) \circ \Phi(f)(y))(m) - (\Phi(f)(y)\circ \Phi(f)(x))(m) \\
                      &= [\Phi(f)(x), \Phi(f)(y)](m)
  \end{align*}
  where the second equality comes from the definition of $ f $ as being a $ \mathfrak{g} $-module action. Hence we have that
  \begin{equation*}
    \Phi(f)([x,y]) = [\Phi(f)(x), \Phi(f)(y)]
  \end{equation*}
  and so $ \Phi(f) $ is a Lie algebra homomorphism.

  Next, let $ f \in \text{Hom}_{k\text{-}\mathbf{LieAlg}}(\mathfrak{g}, \text{Lie}(E)) $. We define $ \Psi(f) $ on pure tensors $ x \otimes m $ by
  \begin{equation}
    \Psi(f)(x \otimes m) = f(x)(m).
  \end{equation}
  This is well defined as $ f(-)(-) $ is bilinear. Moreover, for $ x,y \in \mathfrak{g} $ and $ m \in M $ we have that
  \begin{align*}
    \Psi(f)([x,y] \otimes m) &= f([x,y])(m) \\
                             &= [f(x), f(y)](m) \\
                             &= (f(x)\circ f(y))(m) - (f(y)\circ f(x))(m) \\
                             &= f(x)(f(y)(m)) - f(y)(f(x)(m))
  \end{align*}
  showing that $ \Psi(f): \mathfrak{g} \otimes_k M \to M $ satisfies the criteria for being a $ \mathfrak{g} $-module action.

  The two maps $ \Phi $ and $ \Psi $ are clearly inverses of each other. To see one direction, let $ f \in \mathcal{M}_\mathfrak{g} $, $ x \in \mathfrak{g} $, and $ m \in M $. Then
  \begin{align*}
    (\Psi \circ \Phi)(f)(x \otimes m) &= \Phi(f)(x)(m) \\
                                      &= f(x \otimes m)
  \end{align*}
  showing that
  \begin{equation}
    \Psi \circ \Phi = 1_{\mathcal{M}_{\mathfrak{g}}}.
  \end{equation}
  The other direction is proved similarly.

  We must now show that $ \Phi $ is natural in both $ M $. To this end, let $ \varphi: \mathfrak{g} \to \mathfrak{h} $ be a Lie algebra homomorphism. We must then check commutativity of the following diagram:
  \[\begin{tikzcd}
	  {\mathcal{M}_\mathfrak{g}} & {\text{Hom}_{k\text{-}\mathbf{LieAlg}}(\mathfrak{g}, \text{Lie}(E))} \\
	  {\mathcal{M}_\mathfrak{h}} & {\text{Hom}_{k\text{-}\mathbf{LieAlg}}(\mathfrak{h}, \text{Lie}(E))}
	  \arrow["{\Phi_{\mathfrak{h},M}}"', from=2-1, to=2-2]
	  \arrow["{\Phi_{\mathfrak{g}, M}}", from=1-1, to=1-2]
	  \arrow["{\varphi^*}"', from=2-2, to=1-2]
	  \arrow["{\varphi^*}", from=2-1, to=1-1]
  \end{tikzcd}\]
  Let $ f \in \mathcal{M}_\mathfrak{h} $, $ x \in \mathfrak{g} $, and $ m \in M $. The bottom composition then gives
  \begin{align*}
    (\varphi^* \circ \Phi_{\mathfrak{h}, M})(f)(x)(m) &= \Phi_{\mathfrak{h}, M}(f)(\varphi(x))(m) \\
                                                      &= f(\varphi(x) \otimes m)
  \end{align*}
  while the top composition gives
  \begin{align*}
    (\Phi_{\mathfrak{g}, M} \circ \varphi^*)(f)(x)(m) &= \Phi_{\mathfrak{g}, M}(f \circ (\varphi \otimes \text{id}_M))(x)(m) \\
                                                      &= f(\phi(x) \otimes m)
  .\end{align*}
  Hence the diagram commutes and so $ \Phi $ is natural in $ \mathfrak{g} $.
\end{proof}

\begin{proposition}
  \label{prop:gmodeq}
  There is a natural isomorphism of categories between $ \mathfrak{g} $-\textbf{Mod} and $ U\mathfrak{g} $-\textbf{Mod}.
\end{proposition}
\begin{proof}
  The following is an adaptation of the proof found in Theorem 7.3.3 in \cite{weibel1994homological}.

  Let $ M \in k $-\textbf{Mod} and write $ E = \text{End}_k(M) $. From the adjointness proven in Proposition~\ref{prop:adjointu} we have that
  \begin{equation}
    \text{Hom}_{k\text{-}\mathbf{LieAlg}}(\mathfrak{g}, \text{Lie}(E)) \cong \text{Hom}_{k\text{-}\mathbf{Alg}}(U(\mathfrak{g}), E).
    \label{eq:adjointu}
  \end{equation}
  Combining this with Lemma~\ref{lem:ex7.2.2} we have a bijection
  \begin{equation}
    \Phi:\mathcal{M}_{\mathfrak{g}} \cong \text{Hom}_{k\text{-}\mathbf{Alg}}(U(\mathfrak{g}), E):\Psi.
  \end{equation}
  which is natural in $ \mathfrak{g} $.
  We can use this to define a natural isomorphism
  \begin{equation}
    F:\mathfrak{g}\text{-}\mathbf{Mod} \rightleftarrows U\mathfrak{g}\text{-}\mathbf{Mod}:G.
  \end{equation}
  To see how $ F $ is defined, let $ (M, f) \in \mathfrak{g} $-\textbf{Mod} where $ f: \mathfrak{g} \otimes_k M \to M $ is the defining map of $ M $ as a $ \mathfrak{g} $-module. Then
  \begin{equation}
    F((M, f))\coloneqq (M, \Phi(f))
  .\end{equation}
  In other words, the action of $ x \in U(\mathfrak{g}) $ on $ m \in M $ is given by $ x\cdot m = \Phi(f)(x)(m) $.

  Similarly, to see how $ G $ is defined let $ (M, \varphi) \in U\mathfrak{g} $-\textbf{Mod}. Then
  \begin{equation}
    G((M, \varphi)) \coloneqq (M, \Psi(\varphi)).
  \end{equation}
  Since $ \Phi $ and $ \Psi $ are inverses of each other, the statement follows.
\end{proof}

\begin{corollary}
  \label{cor:projinj}
  The category of $ \mathfrak{g} $-modules has enough projectives and injectives.
\end{corollary}
\begin{proof}
  It is well known that for an arbitrary ring $ R $, the category of left $ R $-modules has enough projectives and injectives (see for example \cite{Monnet2021} or \cite{weibel1994homological}). The statement then follows from Lemma~\ref{lem:eqcat}.
\end{proof}
% subsection $ \mathfrak{g} $-modules (end)
% subsection \mathfrak{g} (end)
% section Lie Algebras and $ \mathfrak{g} $-modules (end)
