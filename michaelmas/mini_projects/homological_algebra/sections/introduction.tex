\section{Introduction} % (fold)
\label{sec:Introduction}
Lie algebras were introduced by the Norwegian mathematician Marius Sophus Lie in the 1870s in order to study the concept of infinitesimal transformations. Although these algebras can be defined purely algebraically, their conceptual origin lies in the study of continuous symmetry groups and differential equations. More formally, one studies the tangent space at the identity of a Lie group (a smooth manifold which can be given a group structure) and relates this to the algebra of vector spaces on the Lie group in order to get a bilinear skew-symmetric map on the tangent space. As the tangent space captures the notion of infinitesimal transformations this is a natural setting in which to study the local structure of the group in a way that is algebraically manageable.

The aim of this mini project is to study the (co)homology of Lie algebras and compute the cohomology in some special cases. In the case of a compact simply connected Lie group $ G $ it turns out---although we will not prove it here---that the de Rham cohomology and Lie algebra cohomology are isomorphic. Thus, Lie algebra cohomology generalizes the concept of de Rham cohomology on compact simply connected Lie groups to all groups which can be given a Lie algebra structure.

For a correspondence between the questions on the mini project sheet and relevant lemmas/propositions/theorems/corollaries see Table~\ref{table:exercises}.

\begin{table}[t]
\centering
\begin{tabular}{ | p{2cm} | p{9cm} | }
\hline
Question No. & Relevant Theorems/Propositions/Corollaries \\
\hline
1 & Proposition~\ref{prop:abelian}\\
\hline
2 & Proposition~\ref{prop:monoidal} \\
\hline
3 & Proposition~\ref{prop:adjointu}\\
\hline
4 & Proposition~\ref{prop:gmodeq}\\
\hline
5 & Corollary~\ref{cor:projinj}\\
\hline
6 & Proposition~\ref{prop:triv} \\
\hline
7 & Corollary~\ref{cor:abelianization} \\
\hline
8 & Theorem~\ref{thm:cheval}\\
\hline
9 & Theorem~\ref{thm:equivalence} \\
\hline
10 & Section~\ref{sec:Computations}\\
\hline
% Add more rows as needed
\end{tabular}
\caption{Questions from the mini project sheet and their relevant references in the text.}
\label{table:exercises}
\end{table}
% section Introduction (end)
