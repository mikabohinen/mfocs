\section{Lie Algebras and $ \mathfrak{g} $-modules} % (fold)
\label{chap:Lie Algebras and Lie Algebra Modules}

\subsection{Basics of Lie Algebras} % (fold)
\label{sec:Basics of Lie Algebras}
\begin{definition}[Lie Algebra]
  A \textbf{Lie algebra} over a field $ k $ is a $ k $-module $ \mathfrak{g} $ equipped with a bilinear skew-symmetric map $ [-,-]: \mathfrak{g} \times \mathfrak{g} \to \mathfrak{g} $ which satisfies the Jacobi identity:
  \begin{equation}
    \forall x,y,z \in \mathfrak{g}\, [x,[y,z]] + [z, [x, y]] + [y, [z, x]] = 0
    \label{eq:jacobi}
  .\end{equation}
  A morphism of Lie algebras $ \phi: \mathfrak{g} \to \mathfrak{h} $ is a $ k $-linear map which respects the bracket, i.e.,
  \begin{equation}
    \phi([x,y]) = [\phi(x), \phi(y)].
    \label{eq:liehom}
  \end{equation}
\end{definition}

For any associative algebra $ A $ there is a natural Lie algebra $ \mathfrak{a}\coloneqq \text{Lie}(A) $ where, for $ x,y \in A $, one defines
\begin{equation*}
  [x,y] = xy - yx
.\end{equation*}

It is straightforward to verify that $ [-,-]: \mathfrak{a} \times \mathfrak{a} \to \mathfrak{a} $ satisfies the Jacobi identity.

\begin{lemma}
  \label{lem:functoriality}
  The assignment $ \text{Lie}:k\text{-}\mathbf{Alg} \to k\text{-}\mathbf{LieAlg} $ as defined above is a functor.
\end{lemma}
\begin{proof}
  Let $ f: A \to B $ be a morphism of $ k $-algebras. The map $ \text{Lie}(f):\text{Lie}(A) \to \text{Lie}(B) $ defined for $ x \in \text{Lie}(A) $ by $ (\text{Lie}(f))(x)=f(x) $ is a Lie algebra morphism. To see this, let $ x,y \in A $, then
  \begin{align*}
    (\text{Lie}(f))([x,y]) &= f([x,y]) \\
                           &= f(xy - yx) \\
                           &= f(xy) - f(yx) \\
                           &= f(x)f(y) - f(y)f(x) \\
                           &= [f(x), f(y)] \\
                           &= [(\text{Lie}(f))(x), (\text{Lie}(f))(y)]
  \end{align*}
  and hence $ \text{Lie}(f) $ is indeed a Lie algebra homomorphism. It trivially holds that $ \text{Lie}(\text{id}_A) = \text{id}_{\text{Lie}(A)} $. Hence the only thing that remains to be checked is that composition is preserved. Let $ f \in \text{Hom}_{k\text{-}\mathbf{Alg}}(A,B) $, $ g \in \text{Hom}_{k\text{-}\mathbf{Alg}}(B,C) $. We then have that
  \begin{align*}
    \text{Lie}(g \circ f) &= g \circ f \\
                          &= \text{Lie}(g) \circ \text{Lie}(f)
  .\end{align*}
\end{proof}

It turns out that Lie has a left adjoint $ U: k\text{-}\mathbf{LieAlg} \to k\text{-}\mathbf{Alg} $ called the universal enveloping algebra. In order to define this functor we must first talk about tensor algebras.
\begin{definition}[Tensor algebra]
  If $ M $ is any $ k $-module, then the \textbf{tensor algebra} $ T(M) $ is the graded associative $ k $-algebra with unit generated by $ M $:
  \begin{equation}
    T(M) \coloneqq k \oplus M \oplus (M \otimes_k M) \oplus \cdots \oplus M^{\otimes n} \oplus \cdots
    \label{eq:gaa}
  \end{equation}
  where $ M^{\otimes n} = \bigotimes_{i = 1}^{n} M $. Multiplication is defined by concatenation of terms in the obvious way.
\end{definition}
We thus see that $ T $ is a functor from the category of $ k $-modules to the category of associative unital $ k $-algebras.
\begin{definition}[Universal enveloping algebra]
  If $ \mathfrak{g} \in k\text{-}\mathbf{LieAlg} $, then the \textbf{universal enveloping algebra} $ U(\mathfrak{g}) $ is the quotient of $ T(\mathfrak{g}) $ by the two sided ideal $ I_\mathfrak{g} $ generated by the relations
  \begin{equation}
    x \otimes y - y \otimes x - [x,y],\quad (x,y \in \mathfrak{g}).
    \label{eq:relations}
  \end{equation}
\end{definition}

\begin{lemma}
  \label{lem:ufunct}
  The construction $ U:k\text{-}\mathbf{LieAlg} \to k\text{-}\mathbf{Alg} $ is functorial.
\end{lemma}
\begin{proof}
  We first note that for $ f \in \text{Hom}_{k\text{-}\mathbf{LieAlg}}(\mathfrak{g}, \mathfrak{h}) $, $ U(f): U(\mathfrak{g}) \to U(\mathfrak{h}) $ and $ x_1, \ldots, x_n \in \mathfrak{g} $ we have that
  \begin{equation}
    U(f)(x_1 \otimes \cdots \otimes x_n) = f(x_1) \otimes \cdots \otimes f(x_n).
    \label{eq:uf}
  \end{equation}
  This is well defined as for $ x,y \in \mathfrak{g} $ we have
  \begin{equation*}
    f(x \otimes y - y \otimes x - [x,y]) = f(x) \otimes f(y) - f(y) \otimes f(x) - [f(x), f(y)] \in I_{\mathfrak{h}}
  \end{equation*}
  where $ I_{\mathfrak{h}} $ denotes the two sided ideal used in the definition of $ U $. Letting $ U(f)|_k = \text{id}_k $ we then have a well defined morphism $ U(f):U(\mathfrak{g}) \to U(\mathfrak{h}) $.

  From Equation~\ref{eq:uf} it follows that
  \begin{equation*}
    U(\text{id}_{\mathfrak{g}}) = \text{id}_{U(\mathfrak{g})}.
  \end{equation*}
  For $ x_1, \ldots, x_n \in \mathfrak{g} $, $ f \in \text{Hom}_{k\text{-}\mathbf{LieAlg}}(\mathfrak{g}, \mathfrak{h})  $, $  g \in \text{Hom}_{k\text{-}\mathbf{LieAlg}}(\mathfrak{h}, \mathfrak{l})  $ we have that
  \begin{align*}
    U(g \circ f)(x_1 \otimes \cdots x_n) &= g(f(x_1)) \otimes \cdots \otimes gf(x_n) \\
                                         &= U(g)(f(x_1) \otimes \cdots \otimes f(x_n)) \\
                                         &= (U(g) \circ U(f))(x_1 \otimes \cdots \otimes x_n)
  \end{align*}
  and hence
  \begin{equation}
    U(g \circ f) = U(g) \circ U(f)
  .\end{equation}
  We thus see that $ U:k\text{-}\mathbf{LieAlg} \to k\text{-}\mathbf{Alg} $ defines a functor.
\end{proof}

\begin{proposition}
  \label{prop:adjointu}
  The functor $  U:k\text{-}\mathbf{LieAlg} \to k\text{-}\mathbf{Alg}  $ is left adjoint to $ \text{Lie}:k\text{-}\mathbf{Alg} \to k\text{-}\mathbf{LieAlg} $.
\end{proposition}
\begin{proof}
  We do this by explicitly constructing the unit and counit. Define the unit $ \eta: \text{Id}_{k\text{-}\mathbf{LieAlg}} \implies \text{Lie}U $ componentwise by the inclusion map $ \iota_\mathfrak{g}: \mathfrak{g} \subset \text{Lie}(U(\mathfrak{g})) $
  \begin{equation}
    \eta_\mathfrak{g} = \iota_\mathfrak{g}
  .\end{equation}
  For the counit $ \epsilon: U\text{Lie} \implies  \text{Id}_{k\text{-}\mathbf{Alg}} $ define it on pure tensors componentwise by multiplication. More specifically
  for $ x_1 \otimes \cdots \otimes x_n \in \text{Lie}(A)^{\otimes n} $ let
  \begin{equation}
    \epsilon_A(x_1 \otimes \cdots \otimes x_n) = x_1\cdots x_n.
  \end{equation}
  Restricted to $ k \subset U(\text{Lie}(A)) $ we let $ \epsilon_A |_k = 0 $. This map is well defined as
  \begin{equation*}
  \epsilon_A(x\otimes y - y \otimes x) = xy - yx = \epsilon_A(xy - yx)
  .\end{equation*}
  Hence we have a well defined $ k $-algebra morphism $ \epsilon_A: U(\text{Lie}(A)) \to A $.

  VERIFY THAT THEY ACTUALLY ARE NATURAL

  To see that $ \eta $ and $ \epsilon $ give the unit and counit of the adjunction we must verify the triangle identities
  \begin{align}
    \epsilon U \circ U \eta &= \text{id}_U \\
    \text{Lie}\,\epsilon \circ \eta\,\text{Lie} &= \text{id}_\text{Lie}
  .\end{align}
  Let $ \mathfrak{g} \in k\text{-}\mathbf{LieAlg} $. It suffices to prove the unit identity on generators of $ U(\mathfrak{g}) $. Let therefore $ x_1, \ldots, x_n \in \mathfrak{g} $. We then have that
  \begin{align*}
    (\epsilon_{U(\mathfrak{g})} \circ U(\eta_{\mathfrak{g}}))(x_1 \otimes \cdots \otimes x_n) &= \epsilon_{U(\mathfrak{g})}(U(\eta_{\mathfrak{g}})(x_1 \otimes \cdots \otimes x_n)) \\
                                                                  &= \epsilon_{U(\mathfrak{g})}(\eta_\mathfrak{g}(x_1) \otimes \cdots \otimes \eta_\mathfrak{g}(x_n)) \\
                                                                  &= \epsilon_{U(\mathfrak{g})}(x_1 \otimes \cdots \otimes x_n) \\
                                                                  &= x_1 \otimes \cdots \otimes x_n \\
                                                                  &= \text{id}_{U(\mathfrak{g})}(x_1 \otimes \cdots \otimes x_n)
  .\end{align*}

  For the counit identity, let $ x \in \text{Lie}(A) $. We then have that
  \begin{align*}
    (\text{Lie}(\epsilon_A) \circ \eta_{\text{Lie}(A)})(x) &= \text{Lie}(\epsilon_A)(x) \\
                                                           &= x \\
                                                           &= \text{id}_{\text{Lie}(A)}
  \end{align*}
  which proves the identity. Hence we see that $ \eta $ and $ \epsilon $ satisfies the requirement for $ U $ being left adjoint to $ \text{Lie} $.
\end{proof}
% subsection Basics of Lie Algebras (end)

\subsection{$ \mathfrak{g} $-modules} % (fold)
\label{sec:gmodules}
Although a Lie algebra $ \mathfrak{g} $ is not a ring there is still a way for $ \mathfrak{g} $-modules to make sense.
\begin{definition}[Left $ \mathfrak{g} $-module]
  Let $ \mathfrak{g} $ be a Lie algebra over the field $ k $. A left $ \mathfrak{g} $-module is a $ k $-module $ M $ together with a bilinear action map $ \mathfrak{g} \otimes_k M \to M $ such that $ x \otimes m $ maps to $ xm $ and
  \begin{equation}
    [x,y]m = x(ym) - y(xm).
    \label{eq:whoknow}
  \end{equation}
  Morphisms of left $ \mathfrak{g} $-modules $ M $ and $ N $ are $ k $-linear maps
  $ f: M \to N $ such that the following diagram commutes
  \[\begin{tikzcd}
	  {\mathfrak{g} \otimes_k M} & {\mathfrak{g} \otimes_k N} \\
	  M & N
	  \arrow[from=1-1, to=2-1]
	  \arrow["f"', from=2-1, to=2-2]
	  \arrow[from=1-2, to=2-2]
	  \arrow["{\text{id}_\mathfrak{g} \otimes f}", from=1-1, to=1-2]
  \end{tikzcd}\]
  where the vertical maps are the action maps.
\end{definition}

\begin{lemma}
  \label{lem:abenriched}
  Let $ \mathfrak{g} $-\textbf{Mod} be the category of $ \mathfrak{g} $-modules. For $ M,N \in \mathfrak{g} $-\textbf{Mod} we have that $ \text{Hom}_{\mathfrak{g}-\mathbf{Mod}}(M, N) \in \mathbf{Ab} $ where $ \mathbf{Ab} $ is the category of abelian groups. Moreover, the composition operator
  \begin{equation*}
  \circ:  \text{Hom}_{\mathfrak{g}-\mathbf{Mod}}(M, N) \times \text{Hom}_{\mathfrak{g}-\mathbf{Mod}}(N, P) \to \text{Hom}_{\mathfrak{g}-\mathbf{Mod}}(M, P)
  \end{equation*}
  is $ \mathbb{Z} $-bilinear.
\end{lemma}
\begin{proof}
  To see that $ \text{Hom}_{\mathfrak{g}-\mathbf{Mod}}(M, N) \in \mathbf{Ab} $ we must define an abelian group structure on $ \text{Hom}_{\mathfrak{g}-\mathbf{Mod}}(M, N) $. This is canonically done by using the underlying abelian group structure of $ N $. In other words, given $ f,g \in   \text{Hom}_{\mathfrak{g}-\mathbf{Mod}}(M, N) $ we define $ f + g $ pointwise:
  \begin{equation}
    (f+g)(m) \coloneqq f(m) + g(m)
    \label{eq:abstr}
  .\end{equation}
  From the abelian group structure of $ N $ we then have that $ f + g = g + f $. We then need to verify that $ f + g $ is a $ \mathfrak{g} $-module morphism. To see this, note that for $ m \in M $ and $ x \in \mathfrak{g} $
  \begin{align*}
    (f + g)(xm) &= f(xm) + g(xm) \\
                &= xf(m) + xg(m) \\
                &= x(f(m) + g(m))
  \end{align*}
  showing commutativity of the following diagram
  \[\begin{tikzcd}
	  {\mathfrak{g} \otimes_k M} & {\mathfrak{g} \otimes_k N} \\
	  M & N
	  \arrow[from=1-1, to=2-1]
	  \arrow["{f+g}"', from=2-1, to=2-2]
	  \arrow[from=1-2, to=2-2]
	  \arrow["{\text{id}_\mathfrak{g} \otimes (f + g)}", from=1-1, to=1-2]
  \end{tikzcd}\]
  and hence $ f+g $ is indeed a $ \mathfrak{g} $-module homomorphism. Thus $ \text{Hom}_{\mathfrak{g}-\mathbf{Mod}}(M, N) \in \mathbf{Ab} $.

  The $ \mathbb{Z} $-bilinearity of $ \circ $ is proven pointwise. Let $ a,b \in \mathbb{Z} $, $ f_1, f_2 \in  \text{Hom}_{\mathfrak{g}-\mathbf{Mod}}(M, N) $, $ g \in \text{Hom}_{\mathfrak{g}-\mathbf{Mod}}(N, P)  $ and $ m \in M $. Then
  \begin{align*}
    ((af_1 + bf_2) \circ g) (m) &= (af_1 + bf_2)(g(m)) \\
                                &= af_1(g(m)) + bf_2(g(m)) \\
                                &= ((af_1)\circ g)(m) + ((af_2)\circ g)(m)
  \end{align*}
  showing linearity in the first coordinate. Linearity in the other coordinate is proven similarly. We thus see that composition is $ \mathbb{Z} $-bilinear.
\end{proof}

\begin{lemma}
  \label{lem:fincoprod}
  The category $ \mathfrak{g} $-\textbf{Mod} admits finite coproducts. Hence it also has a zero object (the empty coproduct).
\end{lemma}
\begin{proof}
  It is enough to show that $ \mathfrak{g} $-\textbf{Mod} has a zero object and binary coproducts. The zero object $ 0 \in k $-\textbf{Mod} works as a zero object in $ \mathfrak{g} $-\textbf{Mod} as well. We thus only need to show that it has binary coproducts.

  To this end, let $ M,N \in \mathfrak{g} $-\textbf{Mod}. Consider the direct sum of $ M $ and $ N $ as $ k $-modules $ M \oplus N $. We will give this a $ \mathfrak{g} $-module structure and show that satisfies the property of being the binary coproduct of $ M $ and $ N $ in the category of $ \mathfrak{g} $-modules. Note that $ \mathfrak{g} \otimes_k (M \oplus N) \cong (\mathfrak{g} \otimes_k M) \oplus (\mathfrak{g} \otimes_k N) $ as $ k $-modules. The two maps $ \mathfrak{g} \otimes_k M \to M $ and $ \mathfrak{g} \otimes_k N \to N $ then combine to give a map
  \begin{equation}
    \mathfrak{g} \otimes_k (M \oplus N) \cong (\mathfrak{g} \otimes_k M) \oplus (\mathfrak{g} \otimes_k N)  \to M \oplus N.
    \label{eq:bincop}
  \end{equation}
  The inclusion maps $ M \to M \oplus N $ and $ N \to M \oplus N $ are then also $ \mathfrak{g} $-module morphisms. The only remaining thing to show then is that $ M \oplus N $ satisfies the universality property. Thus, let $ f: M \to P $ and $ g: N \to P $ be two $ \mathfrak{g} $-module morphisms. Viewing them as $ k $-module morphisms we know there is a $ k $-module morphism $ f \oplus g: M \oplus N \to P $. What remains to show is that it also a $ \mathfrak{g} $-module morphism. In other words, that the following diagram commutes
  \[\begin{tikzcd}
	  {\mathfrak{g} \otimes_k (M \oplus N)} & {\mathfrak{g} \otimes_k P} \\
	  {M\oplus N} & P
	  \arrow[from=1-1, to=2-1]
	  \arrow["{f\oplus g}"', from=2-1, to=2-2]
	  \arrow[from=1-2, to=2-2]
	  \arrow["{\text{id}_\mathfrak{g} \otimes (f \oplus g)}", from=1-1, to=1-2]
  \end{tikzcd}\]
  Pointwise we have that for $ m \in M $, $ n \in N $, and $ x \in \mathfrak{g} $
  \begin{align*}
    (f \oplus g) (x(m+n)) &= (f \oplus g)(xm + xn) \\
                          &= f(xm) + g(xn) \\
                          &= xf(m) + xg(n) \\
                          &= x(f(m) + g(n)) \\
                          &= x((f \oplus g)(m + n))
  \end{align*}
  and hence the diagram does indeed commute which means that $ f \oplus g $ is a $ \mathfrak{g} $-module morphism. Thus $ M \oplus N $ with the imposed $ \mathfrak{g} $-module structure satisfies the universality property and so $ \mathfrak{g} $-\textbf{Mod} has finite coproducts.
\end{proof}

\begin{lemma}
  \label{lem:preabelian}
Let $ M,N \in \mathfrak{g} $-\textbf{Mod}. For all $ f \in \text{Hom}_{\mathfrak{g}\text{-}\mathbf{Mod}}(M, N) $ we have that $ \text{ker}(f) $ and $ \text{coker}(f) $ exists.
\end{lemma}
\begin{proof}
  Let $ f \in \text{Hom}_{\mathfrak{g}\text{-}\mathbf{Mod}}(M, N) $ and let $ \text{ker}(f) $ and $ \text{coker}(f) $ be the kernel and cokernel of $ f $ respectively as a $ k $-module morphism. Suppose $ m \in \text{ker}(f) $ and $ x \in \mathfrak{g} $. Then $ f(xm)=xf(m)=0 $ and hence $ \mathfrak{g} \otimes_k M \to M $ restricts to a map $ \mathfrak{g} \otimes_k \text{ker}(f) \to \text{ker}(f) $ which makes the following diagram commute
  \[\begin{tikzcd}
	  {\mathfrak{g} \otimes_k \text{ker}(f)} & {\mathfrak{g} \otimes_k M} \\
	  {\text{ker}(f)} & M
	  \arrow[from=1-1, to=2-1]
	  \arrow["\iota"', from=2-1, to=2-2]
	  \arrow[from=1-2, to=2-2]
	  \arrow["{\text{id}_\mathfrak{g} \otimes \iota}", from=1-1, to=1-2]
  \end{tikzcd}\]
  where $ \iota $ is the inclusion map. The universality of $ \text{ker}(f) $ as a $ k $-module then gives the same universality of $ \text{ker}(f) $ as a $ \mathfrak{g} $-module which means $ \text{ker}(f) $ indeed is the kernel of $ f $.

  By duality we then have that $ \text{coker}(f) $ as a $ k $-module can also be thought of as a $ \mathfrak{g} $-module and satisfies the criteria for being the cokernel of $ f $.
\end{proof}

\begin{proposition}
  The category $ \mathfrak{g} $-\textbf{Mod} is abelian.
\end{proposition}
\begin{proof}
  From Lemma~\ref{lem:abenriched}--\ref{lem:preabelian} we know that $ \mathfrak{g} $-\textbf{Mod} is pre-abelian. To show that $ \mathfrak{g} $-\textbf{Mod} is abelian we therefore only need to show that every monomorphism is the kernel of its cokernel and vice versa for epimorphisms. Thus, let $ f \in \text{Hom}_{\mathfrak{g}\text{-}\mathbf{Mod}}(M, N) $ be a monomorphism. Considering $ f $ as a $ k $-module homomorphism we know that the statement holds, i.e., that $ f $ is the kernel of the map $ N \to \text{coker}(f) $. FINISH THIS
\end{proof}

NEED TO CHECK THAT $ [x,y]m $ IS A THING FOR ALL THE MAPS

As $ \mathfrak{g} $-\textbf{Mod} is abelian, it makes sense to talk about projective and injective resolutions for objects $ M \in \mathfrak{g} $-\textbf{Mod}. We can then use these resolutions to talk about Ext and Tor which we use to define the (co)homology of $ M $.

However, in order to be able to do this we need that every $ \mathfrak{g} $-module has a projective and injective resolution.
\begin{lemma}
  \label{lem:eqcat}
  If $ F:\mathcal{A} \rightleftarrows \mathcal{B}: G $ is an equivalence of abelian categories then $ \mathcal{A} $ has enough projectives/injectives if and only if $ \mathcal{B} $ has enough projectives/injectives.
\end{lemma}
\begin{proof}
  Let $ F:\mathcal{A} \rightleftarrows \mathcal{B}:G $ be an equivalence of categories. Then $ F $ and $ G $ preserves projective/injective objects. Without loss of generality, assume that $ \mathcal{B} $ has enough projectives and let $ M \in \mathcal{A} $. We want to find a projective object $ P \in \mathcal{A} $ and an epimorphism $ \epsilon: P \to M $. Since $ \mathcal{B} $ has enough projectives we know that there exists a projective object $ P' \in \mathcal{B} $ and an epimorphism $ \epsilon': P' \to F(M) $. We then have that $ G(P') $ is projective in $ \mathcal{A} $ and $ G(\epsilon'):G(P') \to GF(M) $ is an epimorphism. We then use the isomorphism $ \eta_M^{-1}:GF(M) \cong M $ and define $ \epsilon = \eta_M^{-1} \circ \epsilon' $ which is still an epimorphism. Letting $ P = G(P') $ we have an epimorphism $ \epsilon: P \to M $ from a projective object $ P $ to $ M $ showing that $ \mathcal{A} $ has enough projectives.

  The other direction is proved similarly and the case for enough injectives follows by duality.
\end{proof}
\begin{lemma}
  \label{lem:ex7.2.2}
  Let $ E = \text{End}_k(M) $ be the associative $ k $-algebra of $ k $-module endomorphisms of a $ k $-module $ M $. For $ \mathfrak{g} $ a Lie algebra over $ k $ we have the collection of maps $ \mathfrak{g} \otimes_k M \to M $ making $ M $ into a $ \mathfrak{g} $-module is in a natural bijection with Lie algebra homomorphisms $ \mathfrak{g} \to \text{Lie}(E) $. It is in fact natural in both $ M $ and $ \mathfrak{g} $.
\end{lemma}
\begin{proof}
  Let
  \begin{equation}
    \mathcal{M}_\mathfrak{g} = \{f: \mathfrak{g} \otimes_k M \to M \mid f \text{ makes } M \text{ into a } \mathfrak{g}\text{-module}\}
  .\end{equation}
  We want to create a bijection
  \begin{equation*}
    \Phi: \mathcal{M}_{\mathfrak{g}} \rightleftarrows \text{Hom}_{k\text{-}\mathbf{LieAlg}}(\mathfrak{g}, \text{Lie}(E)): \Psi
  .\end{equation*}
  Let $ f \in \mathcal{M}_{\mathfrak{g}} $, $ x \in \mathfrak{g} $, $ m \in M $, and define $ \Phi(f) $ pointwise by
  \begin{equation}
    \Phi(f)(x)(m) = f(x \otimes m) = xm.
  \end{equation}
  To see that this is a Lie algebra homomorphism, let $ x, y \in \mathfrak{g} $, then
  \begin{align*}
    \Phi(f)([x,y])(m) &= f([x,y] \otimes m) \\
                      &= x(ym) - y(xm) \\
                      &= \Phi(f)(x)(\Phi(f)(y)(m)) - \Phi(f)(y)(\Phi(f)(x)(m)) \\
                      &= (\Phi(f)(x) \circ \Phi(f)(y))(m) - (\Phi(f)(y)\circ \Phi(f)(x))(m) \\
                      &= [\Phi(f)(x), \Phi(f)(y)](m)
  \end{align*}
  where the second equality comes from the definition of $ f $ as being a $ \mathfrak{g} $-module action. Hence we have that
  \begin{equation*}
    \Phi(f)([x,y]) = [\Phi(f)(x), \Phi(f)(y)]
  \end{equation*}
  and so $ \Phi(f) $ is a Lie algebra homomorphism.

  Next, let $ f \in \text{Hom}_{k\text{-}\mathbf{LieAlg}}(\mathfrak{g}, \text{Lie}(E)) $. We define $ \Psi(f) $ on pure tensors $ x \otimes m $ by
  \begin{equation}
    \Psi(f)(x \otimes m) = f(x)(m).
  \end{equation}
  This is well defined as $ f(-)(-) $ is bilinear. Moreover, for $ x,y \in \mathfrak{g} $ and $ m \in M $ we have that
  \begin{align*}
    \Psi(f)([x,y] \otimes m) &= f([x,y])(m) \\
                             &= [f(x), f(y)](m) \\
                             &= (f(x)\circ f(y))(m) - (f(y)\circ f(x))(m) \\
                             &= f(x)(f(y)(m)) - f(y)(f(x)(m))
  \end{align*}
  showing that $ \Psi(f): \mathfrak{g} \otimes_k M \to M $ satisfies the criteria for being a $ \mathfrak{g} $-module action.

  The two maps $ \Phi $ and $ \Psi $ are clearly inverses of each other. To see one direction, let $ f \in \mathcal{M}_\mathfrak{g} $, $ x \in \mathfrak{g} $, and $ m \in M $. Then
  \begin{align*}
    (\Psi \circ \Phi)(f)(x \otimes m) &= \Phi(f)(x)(m) \\
                                      &= f(x \otimes m)
  \end{align*}
  showing that
  \begin{equation}
    \Psi \circ \Phi = \text{id}_{\mathcal{M}_{\mathfrak{g}}}.
  \end{equation}
  The other direction is proved similarly.

  We must now show that $ \Phi $ is natural in both $ M $ and $ \mathfrak{g} $. We first prove naturality in $ \mathfrak{g} $. To this end, let $ \varphi: \mathfrak{g} \to \mathfrak{h} $ be a Lie algebra homomorphism. We must then check commutativity of the following diagram:
  \[\begin{tikzcd}
	  {\mathcal{M}_\mathfrak{g}} & {\text{Hom}_{k\text{-}\mathbf{LieAlg}}(\mathfrak{g}, \text{Lie}(E))} \\
	  {\mathcal{M}_\mathfrak{h}} & {\text{Hom}_{k\text{-}\mathbf{LieAlg}}(\mathfrak{h}, \text{Lie}(E))}
	  \arrow["{\Phi_{\mathfrak{h},M}}"', from=2-1, to=2-2]
	  \arrow["{\Phi_{\mathfrak{g}, M}}", from=1-1, to=1-2]
	  \arrow["{\varphi^*}"', from=2-2, to=1-2]
	  \arrow["{\varphi^*}", from=2-1, to=1-1]
  \end{tikzcd}\]
  Let $ f \in \mathcal{M}_\mathfrak{h} $, $ x \in \mathfrak{g} $, and $ m \in M $. The bottom composition then gives
  \begin{align*}
    (\varphi^* \circ \Phi_{\mathfrak{h}, M})(f)(x)(m) &= \Phi_{\mathfrak{h}, M}(f)(\varphi(x))(m) \\
    f(\varphi(x) \otimes m)
  \end{align*}
  while the top composition gives
  \begin{align*}
    (\Phi_{\mathfrak{g}, M} \circ \varphi^*)(f)(x)(m) &= \Phi_{\mathfrak{g}, M}(f \circ (\varphi \otimes \text{id}_M))(x)(m) \\
                                                      &= f(\phi(x) \otimes m)
  .\end{align*}
  Hence the diagram commutes. FINISH THIS
\end{proof}

\begin{proposition}
  \label{prop:gmodeq}
  There is an equivalence of categories between $ \mathfrak{g} $-\textbf{Mod} and $ U\mathfrak{g} $-\textbf{Mod}.
\end{proposition}
\begin{proof}
  The following is an adaptation of the proof found in Theorem 7.3.3 in \cite{weibel1994homological}.

  Let $ M \in k $-\textbf{Mod} and write $ E = \text{End}_k(M) $. From the adjointness proven in Proposition~\ref{prop:adjointu} we have that
  \begin{equation}
    \text{Hom}_{k\text{-}\mathbf{LieAlg}}(\mathfrak{g}, \text{Lie}(E)) \cong \text{Hom}_{k\text{-}\mathbf{Alg}}(U(\mathfrak{g}), E).
    \label{eq:adjointu}
  \end{equation}
  Combining this with Lemma~\ref{lem:ex7.2.2} we have a bijection
  \begin{equation}
    \Phi:\mathcal{M}_{\mathfrak{g}} \cong \text{Hom}_{k\text{-}\mathbf{Alg}}(U(\mathfrak{g}), E):\Psi.
  \end{equation}
  We can use this to define an equivalence
  \begin{equation}
    F:\mathfrak{g}\text{-}\mathbf{Mod} \rightleftarrows U\mathfrak{g}\text{-}\mathbf{Mod}:G.
  \end{equation}
  To see how $ F $ is defined, let $ (M, f) \in \mathfrak{g} $-\textbf{Mod} where $ f: \mathfrak{g} \otimes_k M \to M $ is the defining map of $ M $ as a $ \mathfrak{g} $-module. Then
  \begin{equation}
    F((M, f))\coloneqq (M, \Phi(f))
  .\end{equation}
  In other words, the action of $ x \in U(\mathfrak{g}) $ on $ m \in M $ is given by $ x\cdot m = \Phi(f)(x)(m) $.

  Similarly, to see how $ G $ is defined let $ (M, \varphi) \in U\mathfrak{g} $-\textbf{Mod}. Then
  \begin{equation}
    G((M, \varphi)) \coloneqq (M, \Psi(\varphi)).
  \end{equation}
  Since $ \Phi $ and $ \Psi $ are inverses of each other, the statement follows. REVISE THIS
\end{proof}

\begin{corollary}
  \label{cor:projinj}
  The category of $ \mathfrak{g} $-modules has enough projectives and injectives.
\end{corollary}
\begin{proof}
  It is well known that for an arbitrary ring $ R $, the category of left $ R $-modules has enough projectives and injectives (see for example \cite{Monnet2021} or \cite{weibel1994homological}). The statement then follows from Lemma~\ref{lem:eqcat}.
\end{proof}
% subsection $ \mathfrak{g} $-modules (end)

\subsection{Monoidal structure} % (fold)
\label{sec:Monoidal structure}
MOVE TO APPENDIX
\begin{definition}[Monoidal category]
  A \textbf{monoidal category} is a category $ \mathcal{C} $ with a monoidal structure. A monoidal structure consists of the following data:
  \begin{enumerate}[label=(\roman*)]
    \item a functor
      \begin{equation}
        \otimes: \mathcal{C} \times \mathcal{C} \to \mathcal{C}
        \label{eq:tensor}
      \end{equation}
      from the product category of $ \mathcal{C} $ with itself, called the \textbf{tensor product},
    \item an object $ 1 \in \mathcal{C} $ called the \textbf{unit object} or \textbf{tensor unit},
    \item a natural isomorphism
      \begin{equation}
        a: ((-) \otimes (-)) \otimes (-) \xrightarrow{\cong} (-) \otimes ((-) \otimes (-))
        \label{eq:associator}
      \end{equation}
      with components of the form
      \begin{equation}
        a_{x,y,z}: (x \otimes y) \otimes z \to x \otimes (y \otimes z)
        \label{eq:compformasso}
      \end{equation}
      called the \textbf{associator},
    \item a natural isomorphism
      \begin{equation}
        \lambda: (1 \otimes (-)) \xrightarrow{\cong} (-)
        \label{eq:leftunit}
      \end{equation}
      with components of the form
      \begin{equation}
        \lambda_x: 1 \otimes x \to x
      \end{equation}
      called the \textbf{left unitor}, and
    \item a natural isomorphism
      \begin{equation}
        \rho: (-) \otimes 1 \xrightarrow{\cong} (-)
        \label{eq:rightunit}
      \end{equation}
      with components of the form
      \begin{equation}
      \rho_x: x \otimes 1 \to x
      \end{equation}
      called the \textbf{right unitor},
  \end{enumerate}
  such that the following two diagrams commute, for all objects involved:
  \begin{enumerate}[label=(\roman*)]
    \item the \textbf{triangle identity}
      \[\begin{tikzcd}
	      {(x\otimes 1)\otimes y} && {x \otimes(1\otimes y)} \\
	                              & {x \otimes y}
	                              \arrow["{a_{x,1,y}}", from=1-1, to=1-3]
	                              \arrow["{\rho_x \otimes \text{id}_y}"', from=1-1, to=2-2]
	                              \arrow["{\text{id}_x \otimes \lambda_y}", from=1-3, to=2-2]
      \end{tikzcd}\]
    \item the \textbf{pentagon identity}
      \[\begin{tikzcd}
         & {(w\otimes x) \otimes (y \otimes z)} & \\
	      {w\otimes (x \otimes (y \otimes z))} & & {((w \otimes x)\otimes y \otimes z)} \\
	      {w \otimes ((x \otimes y) \otimes z)} && {(w\otimes (x \otimes y)) \otimes z}
	      \arrow["{\text{id}_w \otimes a_{x,y,z}}"', from=2-1, to=3-1]
	      \arrow["{a_{w,x,y\otimes z}}", from=2-1, to=1-2]
	      \arrow["{a_{w\otimes x,y,z}}", from=1-2, to=2-3]
	      \arrow["{a_{w,x,y}\otimes \text{id}_z}"', from=3-3, to=2-3]
	      \arrow["{a_{w,x\otimes y,z}}"', from=3-1, to=3-3]
      \end{tikzcd}\]
  \end{enumerate}
\end{definition}
\begin{definition}[Symmetric monoidal category]
  A \textbf{symmetric monoidal category} is a monoidal category $ (\mathcal{C}, \otimes, 1) $ with a natural isomorphism
  \begin{equation}
    B: (-) \otimes (-) \to (-) \otimes (-)
    \label{eq:braiding}
  \end{equation}
  with components of the form
  \begin{equation}
    B_{x,y}: x \otimes y \to y \otimes x
  \end{equation}
  called the \textbf{braiding} such that the following diagram commutes, for all objects involved:
  \begin{enumerate}[label=(\roman*)]
    \item the \textbf{hexagon identity}
      \[\begin{tikzcd}
	      {(x \otimes y) \otimes z} & {x\otimes (y\otimes z)} & {(y\otimes z) \otimes x} \\
	      {(y\otimes x)\otimes z} & {y \otimes (x\otimes z)} & {y \otimes (z \otimes x)}
	      \arrow["{B_{x,y}\otimes\text{id}_z}"', from=1-1, to=2-1]
	      \arrow["{a_{x,y,z}}", from=1-1, to=1-2]
	      \arrow["{B_{x,y\otimes z}}", from=1-2, to=1-3]
	      \arrow["{a_{y,z,x}}", from=1-3, to=2-3]
	      \arrow["{a_{y,x,z}}"', from=2-1, to=2-2]
	      \arrow["{\text{id}_y \otimes B_{x,z}}"', from=2-2, to=2-3]
      \end{tikzcd}\]
    \item the \textbf{inverse law}
      \[\begin{tikzcd}
	& {y\otimes x} \\
	      {x\otimes y} && {x\otimes y}
	      \arrow["{B_{x,y}}", from=2-1, to=1-2]
	      \arrow["{B_{y,x}}", from=1-2, to=2-3]
	      \arrow[from=2-1, to=2-3]
	      \arrow["{\text{id}_{x\otimes y}}", from=2-3, to=2-1]
      \end{tikzcd}\]
  \end{enumerate}
\end{definition}

\begin{definition}[Closed category]
  A category $ \mathcal{C} $ is closed if for all $ x,y \in \mathcal{C} $ we have that $ \text{Hom}_{\mathcal{C}}(x,y) \in \mathcal{C} $. In other words the morphisms between any two objects in the category can itself be regarded as an object of that category. TODO
\end{definition}

\begin{definition}[Closed symmetric monoidal category]
  A \textbf{closed symmetric monoidal category} is TODO
\end{definition}
% subsection Monoidal structure (end)

% chapter Lie Algebras and Lie Algebra Modules (end)
