\section{Problem Definition} % (fold)
\label{sec:Problem Definition}
The definitions in this section draw from the concepts outlined in \cite{Christodoulou2016TightBounds} and are further contextualized within the framework of game mechanisms as described in \cite{Syrgkanis2013Composable}. The notation $ [n] $ is used to denote the set $ \{1, \ldots, n\} $.
\subsection{Auctions as mechanisms} % (fold)
\label{sub:Auctions and allocations as mechanisms}
The concept of a ``mechanism'' as described in \cite{Syrgkanis2013Composable} offers a good formal framework to talk about auctions. More specifically,
consider a setting with a set of $ n $ players and outcomes $ \mathcal{X} \subseteq \times_i \mathcal{X}_i $, where each $ \mathcal{X}_i $ is the set of allocations for player $ i \in [n] $ (the possible outcomes for player $ i $). Moreover, each player $ i $ has a valuation function $ v_i: \mathcal{X}_i \to \mathbb{R}_{\geq 0} $ with $ \mathcal{V}_i $ denoting the set of all possible valuations for player $ i $.
\begin{definition}[Mechanism]
  A \textbf{mechanism} $ \mathcal{M} $ is a triple $ \mathcal{M} = (\mathcal{A}, X, P) $, where $ \mathcal{A} = \times_i \mathcal{A}_i $ is a set of actions $ \mathcal{A}_i $ for each player $ i $, $ X: \mathcal{A} \to \mathcal{X} $ is an allocation function that maps each action profile $ a \in \mathcal{A} $ to an outcome $ x \in \mathcal{X} $, and $ P: \mathcal{A} \to \mathbb{R}_{\geq 0}^{n} $ is a payment function that maps each action profile to a payment $ P_i $ for each player $ i $. The \textbf{utility} of player $ i $ for some valuation $ v_i $, allocation $ x_i \in \mathcal{X}_i $, and payment $ p_i = P(x_i) \in \mathbb{R}_{\geq 0} $ is defined to be
\begin{equation}
  u_i^{v_i}(x_i, p_i) = v_i(x_i) - p_i.
  \label{eq:utility}
\end{equation}
\end{definition}
It makes sense to talk about a series of mechanism that run simultaneously and/or sequentially, as in the case of an auction of different items that happens at the same time. We therefore want to consider the more general setting where there are $ n $ players/bidders and $ m $ mechanisms. Each mechanism $ \mathcal{M}_j $ is then a triple $ \mathcal{M}_j = (\mathcal{A}^j, X^j, P^j) $. We make the extra assumption that player $ i $ has a valuation over vectors of outcomes from each mechanism. That is, there is a function $ v_i: \mathcal{X}_i \to \mathbb{R}_{\geq 0} $ for each $ i $ where $ \mathcal{X}_i = \times_j \mathcal{X}_i^j $. The utility of an outcome $ x_i = (x_i^{1},\ldots, x_i^{m}) $ and payment vector $ p_i = (p_i^{1}, \ldots, p_i^{m}) $ for player $ i $ is then given by
\begin{equation}
  u_i^{v_i}(x_i, p_i) = v_i(x_i^{1}, \ldots, x_i^{m}) - \sum_{j = 1}^{m} p_i^{j}.
\end{equation}
\begin{definition}[Simultaneous composition of mechanisms]
  Given $ m $ mechanisms $ \{\mathcal{M}_j\}_{j=1}^m $ as described above with $ n $ bidders/players we define the \textbf{simultaneous composition} of these mechanisms to be the mechanism $ \mathcal{M} = (\mathcal{A}, X, P) $ where $ \mathcal{A}_i = \times_j \mathcal{A}_i^{j} $, $ \mathcal{X} = \times_j \mathcal{X}^{j} $, $ X = \times_j X^{j}: \mathcal{A} \to \mathcal{X} $, and $ P = \sum_{j} P_j: \mathcal{A} \to \mathbb{R}_{\geq 0} $. For $ a \in \mathcal{A} $ we therefore have that $ X(a) = (X^j(a^j))_j $ and $ P(a) = \sum_{j} P^j(a^j) $ with $ a^j $ denoting the $ j $th component of $ a $.
\end{definition}
We also want to have a way to measure the efficiency of an allocation based on some action profile.
\begin{definition}
  Given a mechanism $ \mathcal{M} $ and a valuation profile $ \mathbf{v} \in \times_i \mathcal{V}_i $, the \textbf{social welfare} of an action $ a \in \mathcal{A} $ is defined to be
  \begin{equation}
    SW^\mathbf{v}(a) = \sum_{i = 1}^{n} v_i(X_i(a)).
    \label{eq:social}
  \end{equation}
\end{definition}

Having developed some terminology we can then define the types of mechanisms we are interested in studying.
\begin{definition}[Combinatorial auction]
  Let $ \mathcal{I} $ be a set of items. A \textbf{combinatorial auction} is a mechanism $ \mathcal{M} = (\mathcal{A}, X, P) $ where $ \mathcal{X}_i = 2^\mathcal{I} $ (the powerset of $ \mathcal{I} $), $ \mathcal{X} $ consists of partitions of $ \mathcal{I} $, and $ \mathcal{A}_i = \mathcal{B}_i $ is the set of bids player $ i $ can make. Moreover, each $ v_i \in \mathcal{V}_i $ is assumed to be \textbf{normalized} and \textbf{monotone}, that is $ v_i(\emptyset) = 0 $ and $ S \subseteq T $ $ \implies $ $ v_i(S) \leq v_i(T) $.
\end{definition}
A combinatorial auction is considered to be in the \textbf{full information} setting if for each player $ i \in [n] $ the valuation $ v_i $ is fixed and known to all other players. The auction is considered to be in the \textbf{Bayesian} setting if $ v_i $ is drawn from $ \mathcal{V}_i $ according to some known distribution $ D_i $. The $ D_i $'s are assumed to be independent over the players. In this case the mechanism $ \mathcal{M} = (\mathcal{A}, X, P) $ defines a game of \textbf{incomplete information} and the strategy of each player is a function $ s_i : \mathcal{V}_i \to \mathcal{A}_i $. We will use $ s(\mathbf{v}) = (s_i(v_i))_{i \in [n]} $ to denote the vector of actions given a valuation profile $ \mathbf{v} $ and $ s_{-i}(\mathbf{v}_{-i}) = (s_j(v_j))_{ j \neq i} $ to denote the vector of actions for all players except $ i $. NEED TO REVISE THIS possibly move it up so that we introduce full information and all that before explicitly defining auctions.

\begin{definition}[First-price auction]
  A \textbf{first-price auction} is a combinatorial auction $ \mathcal{M} $ in which the allocation function $ X: \mathcal{A} \to \mathcal{X} $ is described by allocating items to the highest bidder. The price a player must pay for the allocation is equal to the bid if he has the highest bid and zero otherwise.
\end{definition}

A \textbf{simultaneous first-price auction} is then just a simultaneous composition of first-price auctions.


To make the notation less cumbersome we introduce some conventions. Assume players have submitted bids according to bid vectors $ a_i = b_i = (b_{i1}, \ldots, b_{im}) $, and we get a corresponding allocation $ X(\mathbf{b}) $. If $ S \subseteq [m] $ then we let $ b_i(S) = \sum_{j \in S} b_{ij} $. We also use $ v_i(\mathbf{b}) $ and $ SW(\mathbf{b}) $ instead of $ v_i(X_i(\mathbf{b})) $ and $ SW(X(\mathbf{b})) $ if $ X $ is clear from the context. The utility $ u_i $ of player $ i $ in this context is then his valuation of the received set minus his payments: $ u_i(\mathbf{b})=v_i(X_i(\mathbf{b})) - b_i(X_i(\mathbf{b})) $

% subsection Auctions and allocations as mechanisms (end)

\subsection{Nash equilibria and the price of anarchy} % (fold)
\label{sub:Nash equilibria and the price of anarchy}
\begin{definition}[Bidding strategies]
  A \textbf{pure bidding strategy} $ b_i $ for player $ i $ is a vector of bids for the $ m $ items $ b_i = (b_{i1}, \ldots, b_{im}) $. The pure strategy profile for all bidders is then $ \mathbf{b}=(b_1, \ldots, b_n) $. Similarly, a \textbf{mixed strategy} $ B_i $ for player $ i $ is a probability distribution over pure strategies. The mixed strategy profile for all the bidders is then $ \mathbf{B}=(B_1, \ldots, B_n) $.
\end{definition}
We use the notation $ \mathbf{b}_{-i} = (b_1,\ldots, \widehat{b_i}, \ldots, b_n) $ to denote the strategy profile for all players except player $ i $ (the hat denotes a removed item).

As in \cite{Christodoulou2016TightBounds} we state the five relevant notions of equilibria. Let $ \mathbf{v}=(v_1, \ldots, v_n) $ be the players valuations and remember that in the Bayesian setting, $ v_i $ is drawn from $ \mathcal{V}_i $ according to some distribution. Then $ \mathbf{B} $ is called a
\begin{enumerate}[label=(\alph*)]
  \item \textit{pure Nash equilibrium}, if $ \mathbf{B}=\mathbf{b} $ is a pure strategy profile and $ u_i(\mathbf{b}) \geq u_i(b_i', \mathbf{b}_{-i}) $,

  \item \textit{mixed Nash equilibrium}, if $ \mathbf{B} = \times_i B_i $ and $ \mathbb{E}_{\mathbf{b}\sim \mathbf{B}}[u_i(\mathbf{b})] \geq \mathbb{E}_{\mathbf{b}_{-i} \sim \mathbf{B}_{-i}}[u_i(b'_i, \mathbf{b}_{-i})] $,

  \item \textit{correlated Nash equilibrium}, if $  \mathbb{E}_{\mathbf{b}\sim \mathbf{B}}[u_i(\mathbf{b})|b_i] \geq \mathbb{E}_{\mathbf{b} \sim \mathbf{B}}[u_i(b'_i, \mathbf{b}_{-i})|b_i]  $,

  \item \textit{coarse correlated Nash equilibrium}, if $  \mathbb{E}_{\mathbf{b}\sim \mathbf{B}}[u_i(\mathbf{b})] \geq \mathbb{E}_{\mathbf{b} \sim \mathbf{B}}[u_i(b'_i, \mathbf{b}_{-i})]  $,

  \item \textit{Bayesian Nash equilibrium}, if $ B_i(\mathbf{v}) = \times_i B_i(v_i) $ and $$ \mathbb{E}_{\mathbf{v}_{-i}, \mathbf{b}\sim \mathbf{B}(\mathbf{v})}[u_i(\mathbf{b})] \geq \mathbb{E}_{\mathbf{v}_{-i}, \mathbf{b}_{-i} \sim \mathbf{B}_{-i}(\mathbf{v}_{-i})}[u_i(b'_i, \mathbf{b}_{-i})]. $$
\end{enumerate}

Each class in the above list is contained in the next one---where we think of the full information setting as a special case of the Bayesian setting.

\begin{definition}
  Fix an auction $ \mathcal{M} $ and a valuation profile $ \mathbf{v}=(v_1, \ldots, v_n) $. Let $ \mathcal{E} $ be a set of equilibria bidding profiles. The \textbf{price of anarchy} (PoA) for this auction and valuation is then defined to be
  \begin{equation}
    PoA^{\mathcal{M}, \mathbf{v}} = \frac{\max_{\mathbf{b} \in \mathcal{B}} SW^\mathbf{v}(\mathbf{b})}{\min_{\mathbf{b}' \in \mathcal{E}}SW^{\mathbf{v}}(\mathbf{b}')}.
    \label{eq:poa}
  \end{equation}
  Letting $ \mathbb{M} $ denote the set of all auctions of a specific type and $ \mathcal{V}_\mathcal{M} $ the valuation profiles for a given auction $ \mathcal{M} $ we define the price of anarchy for this auction type to be
  \begin{equation}
    PoA = \max_{\mathcal{M} \in \mathbb{M}}\left( \max_{\mathbf{v} \in \mathcal{V}_{\mathcal{M}}} PoA^{\mathcal{M}, \mathbf{v}}\right).
    \label{eq:poagen}
  \end{equation}
\end{definition}

% subsection Nash equilibria and the price of anarchy (end)

\subsection{Submodular and subadditive valuations} % (fold)
\label{sub:Submodular valuations}
We are only interested in situations where for each $ i \in [n] $ the set of valuations $ \mathcal{V}_i $ belongs to a specific class of functions.
\begin{definition}[Submodular valuation]
  A valuation $ v: 2^{[m]} \to \mathbb{R}_{\geq 0} $ is said to be \textbf{submodular} if for $ S,T \subseteq [m] $ we have that
  \begin{equation}
    v(S \cup T) + v(S \cap T) \leq v(S) + v(T)
    \label{eq:submodular}
  .\end{equation}
\end{definition}
\begin{definition}[Subadditive valuation]
  A valuation $ v: 2^{[m]} \to \mathbb{R}_{\geq 0} $ is said to be \textbf{subadditive} if for all $ S,T \subseteq [m] $ we have
  \begin{equation}
    v(S \cup T ) \leq v(S) + v(T).
    \label{eq:subadditive}
  \end{equation}
\end{definition}
It is well known that the set of submodular valuations is contained in the set of subadditive valuations, see for example \cite{doi:10.1137/070680977}.
% subsection Submodular valuations (end)

\subsection{The problem} % (fold)
\label{sub:The problem}
Having set up all necessary notation and language we can now state the problem for submodular and subadditive valuations in the full information setting as well as the Bayesian setting.
\begin{problem}[Submodular case]
  Consider simultaneous first-price auctions with valuation functions in the submodular class. What is the PoA over mixed Nash equilibria and over Bayesian Nash equilibria?
\end{problem}
Similarly, for subadditive valuations we have the corresponding problem:
\begin{problem}[Subadditive case]
  Consider simultaneous first-price auctions with valuation functions in the subadditive class. What is the PoA over mixed Nash equilibria and over Bayesian Nash equilibria?
\end{problem}
We are thus interested in knowing the
% subsection The problem (end)
% section Problem Definition (end)
