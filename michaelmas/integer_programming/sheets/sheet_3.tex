\documentclass[a4paper]{article}
% Some basic packages
\usepackage[utf8]{inputenc}
\usepackage[T1]{fontenc}
\usepackage{textcomp}
\usepackage[english]{babel}
\usepackage{url}
\usepackage{graphicx}
\usepackage{float}
\usepackage{booktabs}
\usepackage{enumitem}
\usepackage{tikz-cd}

\pdfminorversion=7

% Don't indent paragraphs, leave some space between them
\usepackage{parskip}

% Hide page number when page is empty
\usepackage{emptypage}
\usepackage{subcaption}
\usepackage{multicol}
\usepackage{xcolor}

% Math stuff
\usepackage{amsmath, amsfonts, mathtools, amsthm, amssymb}
\usepackage{mathrsfs}
\usepackage{cancel}
\usepackage{bm}
\newcommand\N{\ensuremath{\mathbb{N}}}
\newcommand\R{\ensuremath{\mathbb{R}}}
\newcommand\Z{\ensuremath{\mathbb{Z}}}
\renewcommand\O{\ensuremath{\emptyset}}
\newcommand\Q{\ensuremath{\mathbb{Q}}}
\newcommand\C{\ensuremath{\mathbb{C}}}

\usepackage{systeme}

\let\svlim\lim\def\lim{\svlim\limits}

\let\implies\Rightarrow
\let\impliedby\Leftarrow
\let\iff\Leftrightarrow
\let\epsilon\varepsilon

\usepackage{stmaryrd}
\newcommand\contra{\scalebox{1.5}{$\lightning$}}

\definecolor{correct}{HTML}{009900}
\newcommand\correct[2]{\ensuremath{\:}{\color{red}{#1}}\ensuremath{\to }{\color{correct}{#2}}\ensuremath{\:}}
\newcommand\green[1]{{\color{correct}{#1}}}

\newcommand\hr{
    \noindent\rule[0.5ex]{\linewidth}{0.5pt}
}

\newcommand\hide[1]{}

\usepackage{siunitx}
\sisetup{locale = US}

\usepackage{mdframed}
\mdfsetup{skipabove=1em,skipbelow=0em}
\theoremstyle{definition}
\newmdtheoremenv[nobreak=true]{definition}{Definition}
\newmdtheoremenv[nobreak=true]{property}{Property}
\newmdtheoremenv[nobreak=true]{corollary}{Corollary}
\newmdtheoremenv[nobreak=true]{theorem}{Theorem}
\newmdtheoremenv[nobreak=true]{lemma}{Lemma}
\newmdtheoremenv[nobreak=true]{proposition}{Proposition}

\newtheorem*{example}{Example}
\newtheorem*{notation}{Notation}
\newtheorem*{remark}{Remark}
\newtheorem*{note}{Note}
\newtheorem*{problem}{Problem}
\newtheorem*{observe}{Observe}
\newtheorem*{intuition}{Intuition}
% ... (add other unnumbered theorem-like environments as per your requirement)

\usepackage{etoolbox}
\AtEndEnvironment{example}{\null\hfill$\diamond$}%
% ... (add other environments to end with a diamond if required)

\makeatletter
\def\thm@space@setup{%
  \thm@preskip=\parskip \thm@postskip=0pt
}

\newcommand{\exercise}[1]{%
    \def\@exercise{#1}%
    \subsection*{Exercise #1}
}

\newcommand{\subexercise}[1]{%
    \subsubsection*{Exercise \@exercise.#1}
}

\usepackage{xifthen}
\def\testdateparts#1{\dateparts#1\relax}
\def\dateparts#1 #2 #3 #4 #5\relax{
    \marginpar{\small\textsf{\mbox{#1 #2 #3 #5}}}
}

\def\@lecture{}%
\newcommand{\lecture}[3]{
    \ifthenelse{\isempty{#3}}{%
        \def\@lecture{Lecture #1}%
    }{%
        \def\@lecture{Lecture #1: #3}%
    }%
    \subsection*{\@lecture}
    \marginpar{\small\textsf{\mbox{#2}}}
}

\usepackage{fancyhdr}
\pagestyle{fancy}

\fancyhead[RO,LE]{\@lecture}
\fancyfoot[RO,LE]{\thepage}
\fancyfoot[C]{\leftmark}

\makeatother

\usepackage{todonotes}
\usepackage{tcolorbox}

\tcbuselibrary{breakable}

\newenvironment{correction}{\begin{tcolorbox}[
    arc=0mm,
    colback=white,
    colframe=green!60!black,
    title=Remark,
    fonttitle=\sffamily,
    breakable
]}{\end{tcolorbox}}

\newenvironment{notebox}[1]{\begin{tcolorbox}[
    arc=0mm,
    colback=white,
    colframe=white!60!black,
    title=#1,
    fonttitle=\sffamily,
    breakable
]}{\end{tcolorbox}}

\usepackage{import}
\usepackage{pdfpages}
\usepackage{transparent}
\newcommand{\incfig}[1]{%
    \def\svgwidth{\columnwidth}
    \import{./figures/}{#1.pdf_tex}
}

\pdfsuppresswarningpagegroup=1

\author{Mika Bohinen}

\usepackage{kbordermatrix}
\usepackage{algorithm}
\usepackage{algpseudocode}
\DeclarePairedDelimiter\ceil{\lceil}{\rceil}
\DeclarePairedDelimiter\floor{\lfloor}{\rfloor}

\usepackage{forest}
\usetikzlibrary{shapes.misc, positioning}

\tikzset{
  font=\normalsize,
  tree node/.style = {align=center, inner sep=0pt, draw, circle, minimum size=18},
  tree node label/.style={font=\scriptsize},
  pruned/.style={strike out, draw=red, thick},
  bound label/.style={font=\scriptsize, inner sep=1pt}
}

\forestset{
  declare toks={left branch prefix}{},
  declare toks={right branch prefix}{},
  declare toks={left branch suffix}{},
  declare toks={right branch suffix}{},
  maths branch labels/.style={
    branch label/.style={
      if n=1{
        edge label={node [left, midway] {$\forestoption{left branch prefix}##1\forestoption{left branch suffix}$}},
      }{
        edge label={node [right, midway] {$\forestoption{right branch prefix}##1\forestoption{right branch suffix}$}},
      }
    },
  },
  set branch labels/.style n args=4{%
    left branch prefix={#1},
    left branch suffix={#2},
    right branch prefix={#3},
    right branch suffix={#4},
  },
  branch and bound/.style={
    /tikz/every label/.append style=tree node label,
    maths branch labels,
    for tree={
      tree node,
      math content,
      s sep'+=20mm,
      l sep'+=5mm,
      thick,
      edge+={thick},
    },
    before typesetting nodes={
      for tree={
        split option={content}{:}{content,branch label},
      },
    },
  },
}

% \renewcommand{\kbldelim}{(}
% \renewcommand{\kbrdelim}{)}
\title{Integer Programming \\ Sheet 3}
\begin{document}
  \maketitle
  \textbf{Notation}: We use $ \subset $ to mean inclusion and $ \subsetneqq $ to mean strict inclusion.
  \begin{exercise}{B.1}
    \begin{enumerate}[label=(\roman*)]
      \item We first check that the proposed solution to the LP relaxation given by
        \begin{align*}
          x_j &= 1, \quad\quad\quad\quad\quad\quad j=1, \ldots, r-1 \\
          x_r &=  \frac{b - \sum_{j = 1}^{r -1} a_j}{a_r} \\
          x_j &= 0, \quad\quad\quad\quad\quad\quad j>r
        \end{align*}
        is feasible. To see this, note that
        \begin{align*}
          \sum_{j = 1}^{n} a_jx_j &= \sum_{j = 1}^{r-1} a_j + a_r \frac{b - \sum_{j = 1}^{r - 1} a_j}{a_r} \\
                                  &= b
        \end{align*}
        and hence it is indeed a feasible solution.

        We now construct a dual feasible solution by using complementary slackness. To do this, note first that the LP relaxation is given by
        \begin{align*}
          \max\,\, c^{T}x& \\
          \text{s.t. }  a^{T}x &\leq b \\
          x &\leq 1 \\
          x &\geq 0
        \end{align*}
        and the dual is then
        \begin{align*}
          \min\,\, by_0 + \sum_{i = 1}^{n} y_i& \\
          \text{s.t.}\quad a_i y_0 + y_i &\geq c_i,\quad i = 1,\ldots, n \\
          y &\geq 0
        \end{align*}
        where $ y_0 $ corresponds to the constraint $ a^{T}x \leq b $. Using complementary slackness we can construct a dual optimal solution corresponding to the proposed primal solution.

        The requirement from complementary slackness gives us that we must have
        \begin{align*}
          y_0 &= \frac{c_r}{a_r} \\
          a_iy_0 + y_i &= c_i,\quad i = 1,\ldots r-1 \\
          (1 - x_r)y_r &= 0 \\
          y_i &= 0,\quad i > r.
        \end{align*}
        We therefore have that
        \begin{align*}
          y_0 &= \frac{c_r}{a_r} \\
          y_i &= c_i - a_i \frac{c_r}{a_r},\quad i =1, \ldots, r-1 \\
          y_i &= 0, \quad i \geq r.
        \end{align*}
        This is a feasible solution as, for $ i=1, \ldots, r-1 $
        \begin{align*}
          a_iy_0 + y_i &= a_i \frac{c_r}{a_r} + c_i - a_i \frac{c_r}{a_r} \\
                       &= c_i
        \end{align*}
        and since $ c_i/a_i > c_r /a_r $ for $ i < r $ we have that $ y_i \geq 0 $ for all $ i $. Moreover, for $ i \geq r $ we have, by assumption, that
        \begin{equation*}
          a_i y_0 + y_i = a_i \frac{c_r}{a_r} \geq c_i
        .\end{equation*}
        Hence the proposed dual solution is feasible. Moreover
        \begin{align*}
          by_0 + \sum_{i = 1}^{n} y_i &= b \frac{c_r}{a_r} + \sum_{i = 1}^{r - 1} \left( c_i - a_i \frac{c_r}{a_r} \right) \\
                                      &=\sum_{i = 1}^{r - 1} c_j + c_r\frac{b - \sum_{i = 1}^{r - 1} a_i}{a_r} \\
                                      &= \sum_{j = 1}^{n} c_jx^{*}_j
        .\end{align*}
        which means $ x^{*} $ is optimal.

      \item
        We can use the above result to do the branch-and-bound as we branch on whether to set $ x_r = 1 $ or $ x_r = 0 $. Now, since
        \begin{equation*}
        \frac{17}{5} > \frac{10}{3} > \frac{25}{8} > \frac{17}{4}
        \end{equation*}
        we have that our incumbent is $ x^* = (1, 1, 0, 0) $ and have a lower bound of $ 17+10 = 27 $. Taking the LP-relaxation of $ \mathscr{F} $ (our original problem) we have by the previous part that the optimal solution is given by $ x^{*} = (1, 1, 0.5, 0) $. This gives a dual bound $ \bar{z}=39.5 $. Since all coefficients are integer valued we can round down. The corresponding tree looks like the one in Figure~\ref{fig:bab1}.
        \begin{figure}
          \begin{center}
            \begin{forest}
              branch and bound,
              [\mathscr{F}, name=P0]
              \node[bound label, above right=0pt and 5pt of P0] {39};
              \node[bound label, below right=0pt and 5pt of P0] {27};
            \end{forest}
          \end{center}
          \caption{Step 1 in branch-and-bound.}\label{fig:bab1}
        \end{figure}

        We then branch on whether $ x_3 = 0 $ or $ x_3 = 1 $ and get the tree in Figure~\ref{fig:bab2}.
        \begin{figure}
          \begin{center}
            \begin{forest}
              branch and bound,
              [\mathscr{F}, name=P0
              [\mathscr{F}_1, edge label={node [midway, left] {\(x_3 = 0\)}}, name=P1
              ]
              [\mathscr{F}_2, edge label={node [midway, right] {\(x_3 = 1\)}}, name=P2
              ]
              ]
              \node[bound label, above right=0pt and 5pt of P0] {39};
              \node[bound label, below right=0pt and 5pt of P0] {27};
            \end{forest}
          \end{center}
          \caption{Branching on $ x_3 $.}\label{fig:bab2}
        \end{figure}

        Focusing on $ \mathscr{F}_1 $ we take the LP-relaxation which is given by
        \begin{align*}
          \max\quad 17x_1 + 10x_2 &+ 17x_4 \\
          \text{s.t. } 5x_1 + 3x_2 + 7x_4 &\leq 12 \\
          x_3 = 0 \\
          x &\leq 1 \\
          x &\geq 0
        .\end{align*}
        The optimal solution is then, by part (i), given by $ x^{*}=(1, 1, 0, 4 /7  ) $ yielding the dual bound $ \bar{z}=36.71 $ which we round down to $ \bar{z}=36 $. We then bound on $ x_4 $ and get the tree in Figure~\ref{fig:bab3}.
        \begin{figure}
          \begin{center}
            \begin{forest}
              branch and bound,
              [\mathscr{F}, name=P0
              [\mathscr{F}_1, edge label={node [midway, left] {\(x_3 = 0\)}}, name=P1
              [\mathscr{F}_{11}, edge label={node [midway, left] {\(x_4 = 0\)}}, name=P11]
              [\mathscr{F}_{12}, edge label={node [midway, right] {\(x_4 = 1\)}}, name=P12]]
              [\mathscr{F}_2, edge label={node [midway, right] {\(x_3 = 1\)}}, name=P2
              ]]
              \node[bound label, above right=0pt and 5pt of P1] {36};
              \node[bound label, above right=0pt and 5pt of P0] {36};
              \node[bound label, below right=0pt and 5pt of P0] {27};
            \end{forest}
          \end{center}
          \caption{Branching on $ x_4 $.}\label{fig:bab3}
        \end{figure}
        Taking the LP-relaxation of $ \mathscr{F}_{11} $ we have the LP given by
        \begin{align*}
          \max\quad 17x_1 + 10x_2 & \\
          \text{s.t. } 5x_1 + 3x_2 &\leq 12 \\
          x_3 = 0 \\
          x_4 = 0 \\
          x &\leq 1 \\
          x &\geq 0
        \end{align*}
        which has optimal solution given by $ x^{*}=(1, 1, 0, 0) $ which was the incumbent we started with. We can prune this branch by optimality.

        Taking the LP-relaxation of $ \mathscr{F}_{12} $ gives the LP problem
        \begin{align*}
          \max\quad 17x_1 + 10x_2 + 17& \\
          \text{s.t. } 5x_1 + 3x_2 &\leq 5 \\
          x_3 = 0 \\
          x_4 = 1 \\
          x &\leq 1 \\
          x &\geq 0
        \end{align*}
        which has optimal solution given by $ x^{*}=(1, 0, 0, 1) $ and yields the dual bound $ \bar{z}=34 $. We can then prune this branch by optimality. Our tree now looks like the one in Figure~\ref{fig:bab4}.
        \begin{figure}
          \begin{center}
            \begin{forest}
              branch and bound,
              [\mathscr{F}, name=P0
              [\mathscr{F}_1, edge label={node [midway, left] {\(x_3 = 0\)}}, name=P1
              [\mathscr{F}_{11}, edge label={node [midway, left] {\(x_4 = 0\)}}, name=P11, edge=pruned]
              [\mathscr{F}_{12}, edge label={node [midway, right] {\(x_4 = 1\)}}, name=P12, edge=pruned]]
              [\mathscr{F}_2, edge label={node [midway, right] {\(x_3 = 1\)}}, name=P2
              ]]
              \node[bound label, above right=0pt and 5pt of P11] {27};
              \node[bound label, below right=0pt and 5pt of P11] {27};
              \node[bound label, above right=0pt and 5pt of P12] {34};
              \node[bound label, below right=0pt and 5pt of P12] {34};
              \node[bound label, above right=0pt and 5pt of P1] {34};
              \node[bound label, below right=0pt and 5pt of P1] {34};
              \node[bound label, above right=0pt and 5pt of P0] {36};
              \node[bound label, below right=0pt and 5pt of P0] {27};
            \end{forest}
          \end{center}
          \caption{Pruning and bounding.}\label{fig:bab4}
        \end{figure}

        The next node to look at is $ \mathscr{F}_2 $ which has LP-relaxation given by
        \begin{align*}
          \max\quad 17x_1 + 10x_2 + 25 + 17x_4& \\
          \text{s.t. } 5x_1 + 3x_2 + 7x_4 &\leq 4 \\
          x_3 = 1 \\
          x &\leq 1 \\
          x &\geq 0
        .\end{align*}
        This has optimal solution given by $ x^{*}=(\frac{4}{5}, 0, 1, 0) $ and a dual bound of $ \bar{z} = 38.6$ which we round down to $ \bar{z}=38 $. We then branch on $ x_1 $ and get the tree in Figure~\ref{fig:bab5}.
        \begin{figure}
          \begin{center}
            \begin{forest}
              branch and bound,
              [\mathscr{F}, name=P0
              [\mathscr{F}_1, edge label={node [midway, left] {\(x_3 = 0\)}}, name=P1
              [\mathscr{F}_{11}, edge label={node [midway, left] {\(x_4 = 0\)}}, name=P11, edge=pruned]
              [\mathscr{F}_{12}, edge label={node [midway, right] {\(x_4 = 1\)}}, name=P12, edge=pruned]]
              [\mathscr{F}_2, edge label={node [midway, right] {\(x_3 = 1\)}}, name=P2
              [\mathscr{F}_{21}, edge label={node [midway, left] {\(x_1 = 0\)}}, name=P21]
              [\mathscr{F}_{22}, edge label={node [midway, right] {\(x_1 = 1\)}}, name=P22]]
              ]
              \node[bound label, above right=0pt and 5pt of P11] {27};
              \node[bound label, below right=0pt and 5pt of P11] {27};
              \node[bound label, above right=0pt and 5pt of P12] {34};
              \node[bound label, below right=0pt and 5pt of P12] {34};
              \node[bound label, above right=0pt and 5pt of P2] {38};
              \node[bound label, above right=0pt and 5pt of P1] {34};
              \node[bound label, below right=0pt and 5pt of P1] {34};
              \node[bound label, above right=0pt and 5pt of P0] {36};
              \node[bound label, below right=0pt and 5pt of P0] {27};
            \end{forest}
          \end{center}
          \caption{Branching on $ x_1 $.}\label{fig:bab5}
        \end{figure}
        The LP-relaxation of $ \mathscr{F}_{21} $ is given by
        \begin{align*}
          \max\quad 10x_2 + 25 + 17x_4& \\
          \text{s.t. } 3x_2 + 7x_4  &\leq 4 \\
          x_1 = 0 \\
          x_3 = 1 \\
          x &\leq 1 \\
          x &\geq 0
        \end{align*}
        which has optimal solution given by $ x^{*} = (0, 1, 1, 1 / 7) $, but we see that the solution $ x^{*} = (0, 1, 1, 0) $ is the only integer valued feasible solution in $ \mathscr{F}_{21} $ with dual and primal bound $ \bar{z}=35=\underbar{z} $.

        The node $ \mathscr{F}_{22} $ can be pruned by infeasibility. Hence we end up with the tree in Figure~\ref{fig:bab6}.
        \begin{figure}
          \begin{center}
            \begin{forest}
              branch and bound,
              [\mathscr{F}, name=P0
              [\mathscr{F}_1, edge label={node [midway, left] {\(x_3 = 0\)}}, name=P1, edge=pruned
              [\mathscr{F}_{11}, edge label={node [midway, left] {\(x_4 = 0\)}}, name=P11, edge=pruned]
              [\mathscr{F}_{12}, edge label={node [midway, right] {\(x_4 = 1\)}}, name=P12, edge=pruned]]
              [\mathscr{F}_2, edge label={node [midway, right] {\(x_3 = 1\)}}, name=P2, edge=pruned
              [\mathscr{F}_{21}, edge label={node [midway, left] {\(x_1 = 0\)}}, name=P21, edge=pruned]
              [\mathscr{F}_{22}, edge label={node [midway, right] {\(x_1 = 1\)}}, name=P22, edge=pruned]]
              ]
              \node[bound label, above right=0pt and 5pt of P21] {35};
              \node[bound label, below right=0pt and 5pt of P21] {35};
              \node[bound label, above right=0pt and 5pt of P22] {$ -\infty $};
              \node[bound label, above right=0pt and 5pt of P11] {27};
              \node[bound label, below right=0pt and 5pt of P11] {27};
              \node[bound label, above right=0pt and 5pt of P12] {34};
              \node[bound label, below right=0pt and 5pt of P12] {34};
              \node[bound label, above right=0pt and 5pt of P2] {35};
              \node[bound label, below right=0pt and 5pt of P2] {35};
              \node[bound label, above right=0pt and 5pt of P1] {34};
              \node[bound label, below right=0pt and 5pt of P1] {27};
              \node[bound label, above right=0pt and 5pt of P0] {35};
              \node[bound label, below right=0pt and 5pt of P0] {35};
            \end{forest}
          \end{center}
          \caption{Final tree.}\label{fig:bab6}
        \end{figure}
        We thus see that the optimal solution is $ x^{*}=(0, 1, 1, 0) $ with corresponding optimal value $ z^{*} = 35 $.
    \end{enumerate}
  \end{exercise}

  \begin{exercise}{B.2}
    \begin{enumerate}[label=(\roman*)]
      \item If $ \mathscr{F} = \emptyset $ then there is nothing to prove. Hence, assume that
        \begin{align*}
          x_1^{*} &= \text{argmax}_{x \in \mathscr{F}}(c^{T}x) \\
          x_2^{*} &= \text{argmax}_{x \in \mathscr{F}}(g(x)) \\
          x_3^{*} &= \text{argmax}_{x \in \mathscr{R}}(g(x))
        .\end{align*}
        We then have
        \begin{equation*}
          z = c^{T}x_1^{*} \leq g(x_1^{*}) \leq  g(x_2^{*}) \leq g(x_3^{*}) = w
        \end{equation*}
        as desired.

      \item To show that MW1T is a relaxation of TSP we need to show two things:
        \begin{enumerate}[label=(\arabic*)]
          \item that $ \mathscr{F} \subset \mathscr{R} $ where $ \mathscr{F} $ is the feasible set for TSP and $ \mathscr{R} $ is the feasible set for TSP,
          \item $ c^{T}x \leq g(x) $ where $ g(x) $ is the objective function for the MW1T.
        \end{enumerate}
        Since $ g(x) = c^{T}x $ we see that (2) is trivially true. We then need to show that every solution to the TSP is a 1-tree. However, this is also fairly trivially true as any Hamiltonian circuit is a subgraph of $ G $ which consists of the union of two edges to node 1 and a spanning tree on the rest (all paths are trees). Thus we see that MW1T is a relaxation of TSP.

      \item We propose the following algorithm
        \begin{algorithm}
          \caption{Minimum Weight 1-Tree Algorithm}
          \begin{algorithmic}[1]
            \State \textbf{Input:} A connected graph $G = (V, E)$ with weight function $c: E \rightarrow \mathbb{R}$
            \State \textbf{Output:} A minimum weight 1-tree

            \Procedure{MinWeightOneTree}{$G$}
            \State Choose a node $v_1 \in V$
            \State Find two edges $e_1, e_2 \in E$ adjacent to $v_1$ with minimum weights
            \State $T \gets \{e_1, e_2\}$ \Comment{Initialize the 1-tree with the two minimum edges}
            \State $G' \gets G$ with $ v_1 $ and all edges incident to $ v_1 $ removed
            \State $ T' \gets  $ Apply Kruskal's algorithm to $G'$ to find MST
            \State $T \gets T \cup T'$ \Comment{Combine the edges to form the 1-tree}
            \State \textbf{return} $T$
            \EndProcedure
          \end{algorithmic}
        \end{algorithm}
        Suppose MinWeightOneTree does not produce a minimum weight 1-tree. Then there exists another tree $ T' $ such that, if $ x_e^{1} $ denotes edges corresponding to $ T $ and $ x_e^2 $ denotes edges corresponding to $ T' $, then
        \begin{equation*}
        \sum_{e \in \delta(1)} c_ex^{2}_e + \sum_{e \in E\setminus \delta(1)} c_ex^{2}_e < \sum_{e \in \delta(1)} c_ex^{1}_e + \sum_{e \in E\setminus \delta(1)} c_ex^{1}_e
        .\end{equation*}
        We must then have either
        \begin{equation*}
        \sum_{e \in \delta(1)} c_e x_e^{2} < \sum_{e \in \delta(1)} c_e x_e^{1}
        \end{equation*}
        or
        \begin{equation*}
        \sum_{e \in E \setminus \delta(1)} c_e x_e^{2} < \sum_{e \in E \setminus \delta(1)} c_e x_e^{1}
        .\end{equation*}
        The first option violates the minimality of the edges adjacent to node 1 while the second option violates the optimality of the spanning tree returned by Kruskal's algorithm. Hence we see that the output of MinWeightOneTree must be a minimum 1-tree.

      \item Seeing as a 1-tree has exactly one cycle we can branch by excluding one edge in the cycle. Each subproblem is then of the same type, i.e., a TSP that live in a subgraph of $ G $ and the union of all such subproblems is equal to original problem.
    \end{enumerate}
  \end{exercise}

  \begin{exercise}{B.3}
    \begin{enumerate}[label=(\roman*)]
      \item
        We have that the block matrices $ A_k $ are given by
        \begin{align*}
          A_1 &= \begin{bmatrix}
            1 & 1 \\
            0 & 0 \\
            0 & 0 \\
            0 & 0 \\
            0 & 0
            \end{bmatrix},\quad A_2 = \begin{bmatrix}
            1 & 1 \\
            0 & 0 \\
            1 & 1 \\
            0 & 0 \\
            0 & 0
          \end{bmatrix} \\ \\
            A_3 &= \begin{bmatrix}
              0 & 0 \\
              1 & 1 \\
              0 & 0 \\
              0 & 0 \\
              0 & 0
              \end{bmatrix},\quad A_4 = \begin{bmatrix}
              0 & 0 \\
              1 & 1 \\
              0 & 0 \\
              0 & 0 \\
              1 & 1
            \end{bmatrix} \\ \\
              A_5 &= \begin{bmatrix}
                0 & 0 \\
                1 & 1 \\
                0 & 0 \\
                1 & 1 \\
                0 & 0
                \end{bmatrix},\quad A_6 = \begin{bmatrix}
                0 & 0 \\
                0 & 0 \\
                1 & 1 \\
                0 & 0 \\
                0 & 0
              \end{bmatrix}
      \end{align*}
      \begin{align*}
        A_7 &= \begin{bmatrix}
          0 & 0 \\
          0 & 0 \\
          0 & 0 \\
          0 & 0 \\
          1 & 1
          \end{bmatrix},\quad A_8 = \begin{bmatrix}
          0 & 0 \\
          0 & 0 \\
          1 & 1 \\
          1 & 1 \\
          0 & 0
        \end{bmatrix} \\ \\
        \end{align*}
        We have that the $ D_k $'s which specify the $ \mathscr{X}_k $'s are given by
        \begin{equation*}
          D_k = \begin{bmatrix}
            1 & 0 \\
            \frac{d_k^{2}}{d_k^1 + d_k^2} & 1
          \end{bmatrix}
        \end{equation*}
        so that the $ \mathscr{X}_k $'s are given by
        \begin{equation*}
          \mathscr{X}_k = \left\{ x_k \in \mathbb{Z}^2 \mid \begin{bmatrix}
              1 & 0 \\
              \frac{d_k^2}{d_k^1 + d_k^2} & 1
          \end{bmatrix}x_k \leq \begin{bmatrix}
              d_k^1 + d_k^2 \\
              d_k^{2} + d_k^{3}
          \end{bmatrix}, x_k \geq 0 \right\}
        .\end{equation*}
        Spelled out we then have
        \begin{align*}
          \mathscr{X}_1 &= \left\{ x_1 \in \mathbb{Z}^2 \mid \begin{bmatrix}
              1 & 0 \\
              \frac{1}{3} & 1
          \end{bmatrix} x_1\leq \begin{bmatrix}
            60 \\
            35
          \end{bmatrix}, x_1 \geq 0 \right\} \\
          \mathscr{X}_2 &= \left\{ x_2 \in \mathbb{Z}^2 \mid \begin{bmatrix}
              1 & 0 \\
              \frac{5}{14} & 1
          \end{bmatrix}x_2 \leq \begin{bmatrix}
            70 \\
            45
          \end{bmatrix}, x_2 \geq 0 \right\} \\
          \mathscr{X}_3 &= \left\{ x_3 \in \mathbb{Z}^2 \mid \begin{bmatrix}
              1 & 0 \\
              \frac{3}{11} & 1
          \end{bmatrix}x_3 \leq \begin{bmatrix}
            55 \\
            30
          \end{bmatrix}, x_3 \geq 0 \right\} \\
          \mathscr{X}_4 &= \left\{ x_4 \in \mathbb{Z}^2 \mid \begin{bmatrix}
              1 & 0 \\
              \frac{2}{7} & 1
          \end{bmatrix}x_4 \leq \begin{bmatrix}
            70 \\
            50
          \end{bmatrix}, x_4 \geq 0 \right\} \\
          \mathscr{X}_5 &= \left\{ x_5 \in \mathbb{Z}^2 \mid \begin{bmatrix}
              1 & 0 \\
              \frac{1}{4} & 1
          \end{bmatrix}x_5 \leq \begin{bmatrix}
            60 \\
            30
          \end{bmatrix}, x_5 \geq 0 \right\} \\
          \mathscr{X}_6 &= \left\{ x_6 \in \mathbb{Z}^2 \mid \begin{bmatrix}
              1 & 0 \\
              \frac{3}{8} & 1
          \end{bmatrix}x_6 \leq \begin{bmatrix}
            80 \\
            40
          \end{bmatrix}, x_6 \geq 0 \right\} \\
          \mathscr{X}_7 &= \left\{ x_7 \in \mathbb{Z}^2 \mid \begin{bmatrix}
              1 & 0 \\
              \frac{1}{5} & 1
          \end{bmatrix}x_7 \leq \begin{bmatrix}
            50 \\
            45
          \end{bmatrix}, x_7 \geq 0 \right\} \\
          \mathscr{X}_8 &= \left\{ x_8 \in \mathbb{Z}^2 \mid \begin{bmatrix}
              1 & 0 \\
              \frac{4}{13} & 1
          \end{bmatrix}x_8 \leq \begin{bmatrix}
            65 \\
            40
          \end{bmatrix}, x_8 \geq 0 \right\}
        .\end{align*}
        We can then formulate the revenue maximisation problem of the airline as an IP in block-angular form by
        \begin{align*}
          \text{(RMP) max }\sum_{k = 1}^{8} c_k^{T} x_k & \\
          \text{s.t } \sum_{k = 1}^{8} A_k x_k &\leq \begin{bmatrix}
            100 \\
            200 \\
            150 \\
            100 \\
            100
          \end{bmatrix} \\
            x_k \in \mathscr{X}_k&\text{ for } k=1, \ldots, 8
        .\end{align*}

      \item We write each $ \mathscr{X}_k $ as
        \begin{equation*}
          \mathscr{X}_k = \{x_{k,t} \mid t \in \{1,\ldots, T_k\}\}
        .\end{equation*}
        From this we get the Dantzig-Wolfe reformulation given by
        \begin{align*}
          \text{(RMP)  } \max_{\lambda} \sum_{k = 1}^{8} \sum_{t = 1}^{T_k} (c_k^{T}x_{k,t})\lambda_{k,t}& \\
          \text{s.t. } \sum_{k = 1}^{8} \sum_{t = 1}^{T_k} (A_k x_{k,t})\lambda_{k,t} &\leq \begin{bmatrix}
            100 \\
            200 \\
            150 \\
            100 \\
            100
          \end{bmatrix} \\
            \sum_{t = 1}^{T_k} \lambda_{k,t} &= 1, \quad (k = 1,\ldots, 8) \\
            \lambda_{k,t} \in \{0,1\},& \quad (t=1, \ldots, T_k), \quad (k = 1,\ldots, 8)
        \end{align*}
        which has (LPM) given by
        \begin{align*}
          \text{(LPM)  } \max_{\lambda} \sum_{k = 1}^{8} \sum_{t = 1}^{T_k} (c_k^{T}x_{k,t})\lambda_{k,t}& \\
          \text{s.t. } \sum_{k = 1}^{8} \sum_{t = 1}^{T_k} (A_k x_{k,t})\lambda_{k,t} &\leq \begin{bmatrix}
            100 \\
            200 \\
            150 \\
            100 \\
            100
          \end{bmatrix} \\
            \sum_{t = 1}^{T_k} \lambda_{k,t} &= 1, \quad (k = 1,\ldots, 8) \\
            \lambda_{k,t} \geq 0,& \quad (t=1, \ldots, T_k), \quad (k = 1,\ldots, 8)
        .\end{align*}
        The dual of this (DM) is then given by
        \begin{align*}
          \text{(DM)  } \min_{\mu,\pi} 100 \pi_1 + 200\pi_2 + &150 \pi_3 + 100\pi_4 + 100 \pi_5 + \sum_{k = 1}^{8} \mu_k \\
          \text{s.t. } \pi^{T}A_kx_{k,t} + \mu_k &\geq c_k^{T}x_{k,t},\quad (k = 1,\ldots, 8),\quad (t = 1,\ldots, T_k) \\
          \pi &\geq 0
        .\end{align*}
        The associated column generation subproblems ($\text{CGIP} _k $) are given by
        \begin{align*}
          \text{(CGIP)}_k\quad \zeta_k = \max_{x_k} (c_k - \pi^{T}A_k)x_k - &\mu_k \\
          \text{s.t. } \begin{bmatrix}
            1 & 0 \\
            \frac{d_k^{2}}{d_k^{1} + d_k^2} & 1
            \end{bmatrix} x_k &\leq \begin{bmatrix}
            d_k^{1} + d_k^{2} \\
            d_k^2 + d_k^3
          \end{bmatrix} \\
          x_k &\geq 0 \\
          x_k &\in \mathbb{Z}^2
        .\end{align*}
        Solving the ($ \text{CGIP}_k $) is equivalent to finding $ x_k^{*} = (x_k^{1*}, x_k^{2*}) $ such that if $ c_k - \pi^{T}A_k = (\tilde{c}_k^{1}, \tilde{c}_k^{2})^{T} $ then
        \begin{align*}
          (x_k^{1*}, x_k^{2*}) = \text{argmax}_{x_k} \tilde{c}_k^{1}x_k^{1} &+ \tilde{c}_k^{2}x_k^2 \\
          \text{s.t. } \begin{bmatrix}
            1 & 0 \\
            \frac{d_k^{2}}{d_k^{1} + d_k^2} & 1
            \end{bmatrix} x_k &\leq \begin{bmatrix}
            d_k^{1} + d_k^{2} \\
            d_k^2 + d_k^3
          \end{bmatrix} \\
          x_k &\geq 0 \\
          x_k &\in \mathbb{Z}^2
        .\end{align*}
        There are then four cases to consider
        \begin{enumerate}[label=(\arabic*)]
          \item $ \tilde{c}_k^{1}, \tilde{c}_k^{2} > 0 $: In this case we want both $ x_k^{1} $ and $ x_k^{2} $ to be as large as possible, but because of the constraints there is some interdependence between the two. We thus apply branch and bound in the usual way via LP-relaxation and branching on fractional values.
          \item $ \tilde{c}_k^{1}< 0 $, $ \tilde{c}_k^2 > 0 $: In this case we want $ x_k^{2} $ as large as possible and $ x_k^{1} $ as small as possible. Thus, set $ x_k^{1} = 0 $ and $ x_k^{2} = \floor*{d_k^2 + d_k^3} $.
          \item $ \tilde{c}_k^{1}> 0 $, $ \tilde{c}_k^2 < 0 $: In this case we want $ x_k^{1} $ as large as possible and $ x_k^{2} $ as small as possible. Thus, set $ x_k^{1} = \floor*{d_k^{1} + d_k^2} $ and $ x_k^{2} = 0$.
          \item $ \tilde{c}_k^{1}, \tilde{c}_k^2 < 0 $: Set $ x_k^{1} = 0 = x_k^2 $.
        \end{enumerate}

      \item If we are interested in the LP relaxation (P) of (RMP) then the $ \mathscr{X}_k $ turn into $ \mathscr{R}_k $ where
        \begin{equation*}
          \mathscr{R}_k \coloneqq \left\{ x_k \in \mathbb{R}^2 \mid \begin{bmatrix}
              1 & 0 \\
              \frac{d_k^2}{d_k^1 + d_k^2} & 1
          \end{bmatrix}x_k \leq \begin{bmatrix}
              d_k^1 + d_k^2 \\
              d_k^{2} + d_k^{3}
          \end{bmatrix}, x_k \geq 0 \right\}
        .\end{equation*}
        There is no possible way to enumerate all the points in $ \mathscr{R}_k $, but we can enumerate the vertices of it and then express points inside as convex combinations of the vertices. In other words, by letting $ \{v_{k,t}\}_{t \in \{1, \ldots \mathcal{T}_k \}} $ be an enumeration of the vertices of $ \mathscr{R}_k $, and writing $ x_k = \sum_{t = 1}^{\mathcal{T}_k} \lambda_{k,t}v_{k,t} $ where $ \sum_{t = 1}^{\mathcal{T}_k} \lambda_{k,t} = 1 $ ($ \lambda_{k,t} \geq 0 $), we have the following representation
        \begin{align*}
          \text{(PM)  } \max_{\lambda} \sum_{k = 1}^{8} \sum_{t = 1}^{\mathcal{T}_k} (c_k^{T}v_{k,t})\lambda_{k,t}& \\
          \text{s.t. } \sum_{k = 1}^{8} \sum_{t = 1}^{\mathcal{T}_k} (A_k v_{k,t})\lambda_{k,t} &\leq \begin{bmatrix}
            100 \\
            200 \\
            150 \\
            100 \\
            100
          \end{bmatrix} \\
            \sum_{t = 1}^{\mathcal{T}_k} \lambda_{k,t} &= 1, \quad (k = 1,\ldots, 8) \\
            \lambda_{k,t} \geq 0,& \quad (t=1, \ldots, \mathcal{T}_k), \quad (k = 1,\ldots, 8)
        .\end{align*}
        The important difference between (PM) and (LPM) is that in the (LPM) we consider the convex hull of each $ \mathscr{X}_k $ while for the (PM) we consider a possibly larger set since we first enlarge $ \mathscr{X}_k $ and then consider the vertices of this enlarged set. More concretely, in (PM) we require that $ x_k \in \mathscr{R}_k $ while in (LPM) we require that $ x_k \in \text{conv}(\mathscr{X}_k) $ and we have that
        \begin{equation*}
          \text{conv}(\mathscr{X}_k) \subset \mathscr{R}_k
        .\end{equation*}
        Since the objective function in both cases is the same we see that (PM) is a relaxation of (LPM). Hence we will generally get weaker dual bounds from (PM) as compared to (LPM).
    \end{enumerate}
  \end{exercise}
\end{document}
